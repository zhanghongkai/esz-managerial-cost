\documentclass[11pt]{article}

\usepackage{apjfonts,mathptmx,esvect}
\usepackage{latexsym,amssymb,amsfonts,amsmath}
\usepackage{setspace,array,dcolumn}
\usepackage{accents}
\usepackage{rotating}
\usepackage[margin=1in,nohead]{geometry}
\usepackage{graphicx}
\usepackage[width=.999\textwidth,labelfont=bf]{caption}

\newcommand{\ten}{\mbox{$^{*}$}}
\newcommand{\five}{\mbox{$^{**}$}}
\newcommand{\one}{\mbox{$^{***}$}}
\newcommand{\mc}{\multicolumn}

\newcommand{\Section}[1]{\vspace{-8pt}\section{\hskip -1em.~~#1}\vspace{-3pt}}
\newcommand{\SubSection}[1]{\vspace{-3pt}\subsection{\hskip -1em.~~#1}\vspace{-3pt}}

\newcommand{\Price}{\mbox{\it Price}}
\newcommand{\Cost}{\mbox{\it Cost}}
\newcommand{\Margin}{\mbox{\it Margin}}
\newcommand{\SinceChange}{\mbox{\it SinceChange}}
\newcommand{\Rank}{\mbox{\it Rank}}
\newcommand{\RankOne}{\mbox{\it RankOne}}
\newcommand{\NumBump}{\mbox{\it NumBump}}
\newcommand{\Placement}{\mbox{\it Placement}}
\newcommand{\Density}{\mbox{\it Density}}
\newcommand{\PriceLow}{\mbox{\it PriceLow}}
\newcommand{\PriceRange}{\mbox{\it PriceRange}}
\newcommand{\PriceDisp}{\mbox{\it PriceDisp}}
\newcommand{\AvgMargin}{\mbox{\it AvgMargin}}
\newcommand{\CostTrend}{\mbox{\it CostTrend}}
\newcommand{\CostVol}{\mbox{\it CostVol}}
\newcommand{\FirstPage}{\mbox{\it FirstPage}}
\newcommand{\Present}{\mbox{\it Present}}
\newcommand{\NewFirm}{\mbox{\it NewFirm}}
\newcommand{\Night}{\mbox{\it Night}}
\newcommand{\Weekend}{\mbox{\it Weekend}}
\newcommand{\CostChange}{\mbox{\it CostChange}}
\newcommand{\QuantityBump}{\mbox{\it QuantityBump}}
\newcommand{\AvgRank}{\mbox{\it AvgRank}}
\newcommand{\PriceChange}{\mbox{\it PriceChange}}
\newcommand{\Base}{\mbox{\it Base}}
\newcommand{\Upsell}{\mbox{\it Upsell}}
\newcommand{\Monitor}{\mbox{\it Monitor}}
\newcommand{\Change}{\mbox{\it Change}}

\newcommand{\cut}[1]{C_{\tau}^{\,#1}}

\DeclareMathOperator*{\argmin}{argmin}

\newcolumntype{d}{D{.}{.}{1.3}}
\newcolumntype{p}{D{.}{.}{3.1}}
\newcolumntype{s}{D{.}{.}{3.2}}
\newcolumntype{S}{D{.}{.}{2.1}}

\begin{document}

\begin{spacing}{1.3}

%%%%%%%%%%%%%%%%%%%%%%%%%%%%%%%%%%%%%%%%%%%%%%%%%%%%%%%%%%%%%%%%%%%%%%%

\begin{titlepage}

\singlespace

\begin{center}
\large \bf
Costs of Managerial Attention and Activity as a Source of Sticky Prices:\\
Structural Estimates from an Online Market
\end{center}

\vfill

\begin{center}
\begin{tabular*}{\textwidth}{c @{\extracolsep{\fill}}  c c}
Sara Fisher Ellison & Christopher M. Snyder & Hongkai Zhang\\
{\em M.I.T. and CESifo} & {\em Dartmouth College and NBER} & {\em Pandora Media Inc.}
\end{tabular*}
\end{center}

\vfill

\begin{center}
May 2018
\end{center}

\vfill

\noindent {\bf Abstract:} We study price dynamics for computer
components sold on a price-comparison website. Our fine-grained
data---a year of hourly price data for scores of rival
retailers---allow us to estimate a dynamic model of competition,
backing out structural estimates of managerial frictions. The
estimated frictions are substantial, concentrated in the act of
monitoring market conditions rather than entering a new price. We use
our model to simulate the counterfactual gains from automated price
setting and other managerial changes. Coupled with supporting
reduced-form statistical evidence, our analysis provides a window into
the process of managerial price setting and the microfoundation of
pricing inertia, issues of growing interest in industrial organization
and macroeconomics.

\vfill

\noindent {\bf Keywords:} price dynamics, managerial costs, behavioral
economics, structural estimation

\vfill

\noindent {\bf JEL Codes:} L11 (Production, Pricing, and Market
Structure); C73 (Stochastic and Dynamic Games), D22 (Firm Behavior:
Empirical Analysis), L81 (e-Commerce)

\vfill

\noindent {\bf Contact Information:} Ellison: Department of Economics,
M.I.T., 50 Memorial Drive, Cambridge, MA 02142; email
sellison@mit.edu.  Snyder: Department of Economics, Dartmouth College,
301 Rockefeller Hall, Hanover, NH 03755; email
chris.snyder@dartmouth.edu. Zhang: Pandora Media Inc., 2100 Franklin
Street, Oakland CA 94612; email hongkaicol@gmail.com.

\vfill

\noindent {\bf Acknowledgments:} The authors are grateful to Michael
Baye, Steve Berry, Judy Chevalier, Oliver Compte, Glenn Ellison, Jakub
Kastl, Peter Klenow, John Leahy, Emi Nakamura, Whitney Newey, Mo Xiao,
and Jidong Zhou for their insightful advice; to Masao Fukui for
outstanding research assistance; and to seminar participants at
Bologna, Cornell, Harvard, M.I.T., Paris School of Economics, Toronto,
and Yale and conference participants at the ASSA Meetings, HSE--Perm
International Conference on Applied Research in Economics,
International Industrial Organization Conference (Boston),
International Symposium on Recent Developments in Econometric Theory
with Applications in Honor of Jerry Hausman (Xiamen University), NBER
Summer Institute (IT and Digitization and Economic Fluctuations
Meetings), and Tuck School of Business Winter Industrial Organization
Workshop for helpful comments.

\end{titlepage}

%%%%%%%%%%%%%%%%%%%%%%%%%%%%%%%%%%%%%%%%%%%%%%%%%%%%%%%%%%%%%%%%%%%%%%%
\newpage

\Section{Introduction}
\label{s:Intro}

Managerial frictions can be broadly construed as a cost associated
with the manager's overlooked but essential input into the operation
of a firm. Textbook industrial-organization models leave little room
for managerial frictions. Consider, for example, the ubiquitous
pricing decision. Firms in most industries have some discretion over
their output price, which may require dynamic adjustment to changing
supply and demand conditions. When does the manager make these price
changes? By how much? Textbook industrial-organization models would
say price is changed precisely when and by the amount dictated by
strategic considerations based on all available information. Other
economic subdisciplines have paid more attention to the frictions in
this pricing process. Macroeconomists have developed a class of models
in which price inertia plays a key role in the monetary transmission
mechanism, leading to a large theoretical and empirical macro
literature studying a range of frictions such as menu costs, the costs
of monitoring rivals' prices, managerial inattention, and the implicit
contracts that firms have with their customers to maintain prices.
Organizational economists have long been interested in the internal
processes that firms use to overcome the complexities associated with
price setting in practice, dating back to Simon's (1962) claim that
``Price setting involves an enormous burden of information gathering
and computation that precludes the use of any but simple rules of
thumb as guiding principles,'' and to Cyert and March's (1963) classic
book documenting the rule of thumb used by a department store to price
its merchandise.

In this paper, we help bridge the gap between these literatures by
extending methods used in industrial organization to estimate dynamic
models to provide some of the first structural estimates of the sort
of managerial frictions posited by macro and organizational
economists.  We study the pricing decisions of a set of rival firms
selling computer components in an online marketplace, Pricewatch,
which ranks the lowest-price firms in the most prominent positions,
generating the highest sales. Price changes were relatively frequent
in this market (the average spell between price changes lasting about
a week) yet far from continuous.  Initial reduced-form analysis
coupled with information from manager interviews provide evidence that
managerial frictions played a role in this price inertia, motivating
us to propose a structural model to quantify these frictions.  Our
fine-grained data (hourly price observations for scores of firms over
a year) combined with fluctuating market conditions and the jockeying
for price rank generate a rich sample of price changes, an ideal
environment to estimate a dynamic model of competitive price
adjustment. We use the model to back out structural estimates of
managerial frictions, separately identifying a manager's cost of
monitoring the market from the menu cost he or she faces when entering
a new price. These structural estimates provide insight into the
microfoundations of the observed price inertia, a perennial concern of
macro and organizational economists, but also of interest to
industrial-organization economists concerned with price dynamics.

We estimate our dynamic, structural model following the two-step
method of Bajari, Benkard, and Levin (2007) (BBL) for estimating
structural parameters in the profit function. The first step involves
estimating reduced-form strategies constituting a Markov-perfect
equilibrium. The second step involves finding structural parameters in
the profit function that rationalize the estimated strategies such
that no player can improve his payoff by deviating to some other
strategy. To allow for feasible estimation in our complex setting and
to accommodate the possibility of boundedly rational managers---a
natural possibility in our setting in which managerial frictions are
the focus---we make several modifications to BBL's procedure.  The
reduced-form policy functions estimated in the first step are not
required to constitute a Markov-perfect equilibrium based on the
universe of state variables but rather are taken to be rule-of-thumb
strategies based on a subset of salient ones. In the second stage, we
search for structural parameters rationalizing the estimated rule of
thumb as undominated by any deviation in the class of admissible rules
of thumb. Rather than requiring that the firm's strategy to be optimal
in every state, we impose the weaker requirement that the firm chooses
a rule of thumb ex ante maximizing its expected present discounted
value of the flow of surplus over the play of the game, where the
expectation is taken over the distribution of salient states
experienced by the firm.  The computational burden that the full state
space would impose is daunting. Just considering permutations of the
scores of sample firms across price ranks, putting aside many other
payoff-relevant variables (costs, margins, distances between firms in
price space, etc.), there are billions times more such permutations in
a single period than grains of sand on earth.  Even if estimation were
feasible for the econometrician, it may be unrealistic to assume that
managers could compute the best responses each instant conditional on
such a complex state space, especially in our setting in which the
simple act of monitoring rivals' prices will present a substantial
managerial friction, to say nothing of performing wildly complicated
calculations.  Our approach---as one of the first attempts to
accommodate boundedly rational players in a dynamic structural
model---presents its own challenges, to be successful requiring the
specification of a parsimonious policy functions providing a good fit
to observed play. We will thus need to pay considerable attention to
the specification of the functional form and covariates included in
the policy function and to the measurement of the policy function's
goodness of fit.  To improve the fit of the policy functions, we
modify the treatment of firm heterogeneity in BBL's framework. Instead
of estimating the mean, variance, and other parameters characterizing
the continuous distribution of structural parameters across firms, we
postulate discrete types of firms, use machine-learning techniques to
partition firms into the types ex ante, and estimate separate
structural parameters for each type. They key benefit of assuming
several discrete types is that we can improve the flexibility of the
policy-function specification by allowing its coefficients to vary
freely across types. For parsimony, we end up postulating three types;
if we instead try the structural estimation with a single policy
function, its fit and the structural estimates are quite poor.

Consistent with recent work microfounding price stickiness, including
contributions to the macro theory literature by Alvarez, Lippi, and
Paciello (2011) and Bonomo {\em et al.} (2015), we build two separate
managerial frictions into the structural model, (a)~a cost of
monitoring market conditions to determine the appropriate timing and
size of a price change and (b)~a menu cost, i.e., the physical cost of
recording a new price. Though we do not observe monitoring, we are
still able to identify the cost of this latent action through the
exclusion restriction that certain payoff-relevant variables are only
available to the manager after monitoring. We find substantial
monitoring costs---around \$68 per episode for the most prominent firm
type---but virtually no menu costs. The division of managerial
frictions matters for firm profit. In the counterfactual exercise in
which these costs are reversed, with costless monitoring and a \$68
menu cost, the firm's net profit rises substantially because the firm
is able to tailor its price policy more closely to market
conditions. In another counterfactual exercise, we document
substantial profit gains from an increasingly automated price-setting
policy (which firms went on to implement after our sample period).
Our structural estimates confirm the findings of Alvarez, Lippi, and
Paciello (2015), who use survey data to calibrate their previously
cited dual-cost model, finding that the monitoring cost is three times
the menu cost of changing price.

A limit of our approach is that our conclusions are derived from a
small and specialized market, not a broader set of markets
representative of the U.S. economy.  But even then, we would argue
that more general lessons can be drawn from our findings here, and the
detail of the results more than outweighs the disadvantage of
specificity.  For instance, measuring price stickiness in an online
marketplace is a useful complement to much of the existing empirical
evidence, which so far has concentrated on brick-and-mortar
stores.\footnote{Exceptions include Arbatskaya and Baye (2004),
  Chakrabarti and Scholnick (2005), L\"unneman and Wintr (2006),
  Gorodnichenko and Weber (2015), and the ``Billion Prices Project''
  (Cavallo and Rigobon 2012) in progress.}  Also, we would argue that
simply understanding the price-setting mechanisms in one market very
well, as we can with our unusually detailed data, can help us predict
which factors are likely to be important in many other
markets. Finally, structural estimation would be difficult without the
fine-grained, market-specific data provided by our setting. Our
structural approach enables us to simulate counterfactual managerial
costs for a firm, including higher costs that the manager of a
brick-and-mortar store might face.

The paper is structured as follows. Section~\ref{s:Literature} reviews
related literatures. Section~\ref{s:Setting} provides background on
the Pricewatch market. Section~\ref{s:Data} describes the
data. Section~\ref{s:Reduced} highlights several important
reduced-form results from the data. Besides motivating the setup of
the structural model, these reduced-form results contribute
independent insights into price-changing behavior, which have the
appeal of being non-technical and fairly assumption-free.  The rest of
the paper is devoted to specification and estimation of the structural
model. Section~\ref{s:Model} specifies the three components of our
structural model: the value, profit, and policy
functions. Section~\ref{s:Policy} presents estimates of the policy
function and assesses the goodness of its
fit. Section~\ref{s:Structural} discusses identification of the
structural parameters, presents the structural estimates, and provides
a sensitivity analysis.  Section~\ref{s:Counterfactuals} conducts
counterfactual simulations showing how a firm's pricing behavior
changes when managerial costs are varied from the estimated
levels. The last section concludes. An appendix provides technical
details on the structural estimation.

%%%%%%%%%%%%%%%%%%%%%%%%%%%%%%%%%%%%%%%%%%%%%%%%%%%%%%%%%%%%%%%%%%%%%%%

\Section{Literature Review}
\label{s:Literature}

In this section, we provide more detail on where our paper fits in the
literatures of several subfields. Methodologically, our paper lies in
the industrial-organization literature, in particular the literature
that structurally estimates dynamic games, including seminal papers by
Aguirregabiria and Mira (2007); Pakes, Ostrovsky, and Berry (2007);
Pakes, {\em et al.}\ (2015); and BBL cited in the introduction. Our
estimation strategy is based on BBL, modified to allow boundedly
rational managers to choose a rule-of-thumb policy. Our paper studies
the same Pricewatch market and uses the same data as Ellison and
Ellison (2009a, 2009b). Those papers focus on the demand side,
estimating demand elasticities with respect to price, rank, avoided
sales tax, and geographic proximity between buyer and seller. We
shifts the focus to the supply side, analyzing pricing dynamics,
allowing for the strategic interaction among firms, and structurally
estimating managerial frictions.  We will take the demand estimates
from Ellison and Ellison (2009a) as an input into our analysis.

Our paper touches on several other strands of the
industrial-organization literature. The field has a distinguished
history of interest in price rigidity, dating at least back to
Gardiner Means's testimony to Congress testimony about inflexibility
of prices during the Depression (Means 1935, documented in Terkel
1970). Stigler and Kindahl (1970) and Carlton (1986) bypass the
Consumer Price Index (CPI) computed by the Bureau of Labor Statistics
(BLS) and indices computed by other administrative agencies, drilling
into the underlying transactions data to establish a body of stylized
facts about price rigidity.

Price rigidity is but one issue under the broader heading of dynamic
pricing. Another issue that has received attention among empirical
industrial-organization economists is documenting the occurrence of
Edgeworth cycles (Edgeworth 1925, formalized by Maskin and Tirole
1988), in which firms gradually undercut each other until they reach
the zero-profit level where they stay until one relents, resulting in
a dramatic price rise.  Edelman and Ostrovsky (2010) and Zhang (2010)
document Edgeworth cycles in sponsored-search and online-advertising
auctions, respectively.\footnote{A number of papers attempt to
  characterize Edgeworth cycles and other interesting pricing dynamics
  in gasoline markets, including Castanias and Johnson (1993); Eckert
  and West (2004); Noel (2007a, 2007b, 2008); Hosken, McMillan, and
  Taylor (2008); Atkinson (2009); Lewis (2009); Wang (2009); and
  Doyle, Muehlegger, and Samphantharak (2010).} Our setting resembles
these Internet auctions in that firms continuously ``bid'' for
favorable rank positions by cutting their margins---indeed winning the
rank-1 position often requires the firm to have a negative margin of
price over cost.  Our reduced-form analysis will document cycles in
ranks, but in reverse of the Edgeworth pattern: a given firm's rank
gradually rises as others undercut as they adjust their prices for
secular declines in cost, punctuated by sharp drops in the given
firm's rank when the manager attends to pricing, readjusting to a
target rank.

A large industrial-organization literature documents price dispersion
in homogeneous, consumer-good markets ranging from retail gasoline
(Barron, Taylor, and Umbeck 2004; Hosken, McMillan, and Taylor 2008;
Lewis 2008) to books (Clay, {\em et al}.\ 2002) to general retail
(Lach 2002). Closest to our setting is Baye, Morgan, and Scholten
(2004), who document price dispersion in an price-comparison website
using rich data collected with a web-scraping technology. Although we
also document price dispersion in our reduced-form analysis, this is
not one of our core results, which are structural estimates of
managerial frictions. However, our core results can be viewed as
contributing to the price-dispersion literature.  Our finding that
substantial managerial frictions contribute to price stickiness
provides a novel explanation for equilibrium price dispersion.  Most
explanations rely on costly price search \`a la Stahl (1989), but
managerial frictions could also generate a distribution of prices for
homogeneous goods in equilibrium.  That is, our structural estimates
raise the possibility that supply-side rather than demand-side
frictions drive price dispersion.

Our paper is related to several subfields outside of industrial
organization. Substantively, our analysis is related to a large
literature in macroeconomics on sticky prices, including notable
papers by Blinder {\em et al.}\ (1998); Bils and Klenow (2004);
Nakamura and Steinsson (2008); and Eichenbaum, Jaimovich, and Rebelo
(2011). As summarized in Klenow and Malin's (2011) handbook chapter,
this literature uses a range of different data sources from manager
surveys to supermarket scanner data to price data used by the BLS to
compute the CPI. These are typically reduced-form studies documenting
facts about frequency and size of price changes as well as identifying
sectors with the most price rigidity (nondurable goods, processed
goods, goods with noncyclical demand, and goods sold in more
concentrated markets).  Comparing our methodology to this
literature's, we also document descriptive facts in our from an
initial reduced-form analysis, but our main goal is structural
estimation of managerial frictions, which this literature does not
undertake.  Comparing our data to this literature's, each has certain
advantages.  The macro studies tend to use comprehensive datasets of
retail prices, representative of the economy, whereas we look at a
single product. The tradeoff is that their datasets have less detail
for each product. For example, BLS price data is collected monthly, so
price movements within a month are not recorded. We have hourly data,
and we do, in fact, observe many price changes within a month. We
observe prices for most of the rivals operating in the market,
allowing us to estimate firms' reactions to their rivals' price
movements.  The BLS does not attempt to sample all the substitutes in
a product space. The nature of our market and the length of our time
series results in our recording dozens of price changes for each firm,
allowing us to estimate a detailed policy function for price changes
at the firm level. We also have unusually detailed auxiliary data on
wholesale costs and quantities, integral to the structural
estimation.\footnote{A recent addition to the macro literature, a
  working paper by Gorodnichenko and Weber (2015), studies pricing in
  an online platform serving many retailers, generating data of
  similar richness as ours. Like most of this literature, the authors
  do not estimate parameters governing firm interactions or managerial
  decisions, so, although their empirical setting is similar to ours,
  their goals and methods are quite different. A study of the market
  for used cars in Germany in Artinger and Gigerenzer's (2012) working
  paper spans several literatures, macro as well as organizational
  economics. In the spirit of the macro literature, in particular
  Blinder {\em et al.}\ (1998), the first part of their paper presents
  descriptive evidence from extensive manager surveys. The second part
  of their paper gathers online price data from hundreds of car
  dealers to test Simon's (1955) model of aspirational pricing. While
  we share their goal of understanding what drives firms to change
  online prices, the direction of our analysis is quite different,
  estimating a model of price changing by firms and structural
  parameters capturing managerial frictions.}

Ours is closest to a suite of papers from macro as well as empirical
trade that back out managerial frictions in price setting by, in
effect, measuring the imperfect pass through of input-price shocks
(generated by exchange-rate or commodity-price movements, depending on
the study) to product prices.  Two pioneering studies estimate models
of price setting by a single agent with costly adjustment, Slade
(1998) in the market for crackers sold at grocery stores, Davis and
Hamilton (2004) in the market for wholesale gasoline.  More recently,
Nakamura and Zerom (2010) and Goldberg and Hellerstein (2013) move
beyond single-agent models to oligopolies, using a static oligopoly
model including menu costs. Nakamura and Zerom (2010) find a menu cost
of \$7,000 in their calibration exercise, far greater than any
managerial friction we estimate, though their figure is still a small
percentage (less than 0.25\%) of revenue. Goldberg and Hellerstein
(2013) estimate a menu cost of \$60 to \$230, or 1\% to 3\% of revenue
in their structural estimation of the static oligopoly model, in the
range of our estimated monitoring cost, considerably higher than the
negligible menu cost we found in our online setting.  Similar to these
previous papers, we use variation in input prices, in our case the
wholesale price of memory chips, to provide crucial exogenous
variation behind our estimation of managerial frictions. We diverge
from these seminal papers in two ways, by incorporating the dynamics
of firm interactions and by distinguishing two types of managerial
costs, a crucial distinction, given that we find menu costs to be
dominated by a different form of managerial friction, monitoring
costs.

%%%%%%%%%%%%%%%%%%%%%%%%%%%%%%%%%%%%%%%%%%%%%%%%%%%%%%%%%%%%%%%%%%%%%%%

\Section{Empirical Setting}
\label{s:Setting}

Our empirical setting is the online marketplace for computer
components mediated by Pricewatch, previously studied by Ellison and
Ellison (2009a, 2009b); see those papers for additional details on the
empirical setting. During the 2000--01 period during which our data
were collected, the Pricewatch marketplace was composed of a large
number of small, undifferentiated e-retailers selling memory upgrades,
CPUs, and other computer parts.  These retailers tended to run
bare-boned operations with spartan offices, little or no advertising,
rudimentary websites, and no venture capital. A large fraction of
their customers came through Pricewatch. Instead of click-through
fees, retailers paid Pricewatch a monthly fee to list
products. Customers could use Pricewatch to locate a product in one of
two ways, either typing a product description into a search box or
running through a multi-layered menu to select one of a number of
predefined product categories. For example, clicking on ``System
Memory'' and then on ``PC100 128MB SDRAM DIMM'' would return a list of
products in that category sorted from cheapest to most expensive in a
format with twelve listings per page. The full list for a pre-defined
category could span dozens of pages with listings from scores of
retailers. Figure~\ref{f:pricewatch} contains the first page of a
typical list for the memory module in our study, downloaded during our
sample period.

The Pricewatch ranking exhibited substantial churn from day to day and
even from hour to hour. Figure~\ref{f:series} illustrates the movement
in prices (top row of panels) and ranks (bottom row of panels) for
three representative retailers of PC100 128MB memory modules in our
sample during a representative month. The first and third retailers
changed price six times during the month, the second retailer
twice. Even the second firm's slower rate of price change is quite
rapid compared to findings in the macro literature from U.S. Bureau of
Labor Statistics data that the median length of spells between price
changes of 4.3 months (Bils and Klenow, 2004). While frequent,
retailers' price changes were still far from continuous. When firms
were not changing prices, their ranks continued to move, bumped up or
down by rival price changes, leading a firm's rank to fluctuate much
more than its price.

Based on information from a detailed interview of a manager of one of
the retailers participating on Pricewatch, we can identify several
possible reasons for this churn in the rankings. First, wholesale
prices could be quite volatile for some products. Retailers would
receive wholesale-price quotes via daily emails, and they often
fluctuated from day to day. Short-term fluctuations could be up or
down, but as these are electronic components, the long-term trend was
downward. The daily price quotes were relevant to the manager's
operations as they typically carried little or no inventory, ordering
enough to cover just the sales since the previous day's order. A
second reason for turnover in the rankings was that managers did not
continuously monitor each individual product's rank on the Pricewatch
website, making the instantaneous changes needed to maintain a
particular rank. Retailers typically offered scores of products in
different categories on Pricewatch; it would be impractical for a
manager to continuously monitor all of them even he or she attended to
no other managerial tasks. Certain high-volume products, such as the
specific memory modules we study, could merit more attention, but this
might involve checking at most one or two times a day. During our
study period, managers had to enter price changes manually into
Pricewatch's database, as automated price setting was not introduced
until 2002. Each adjustment would thus involve a fixed cost in both
determining and entering the appropriate price.

A retailer's rank on Pricewatch was a key determinant of its sales and
profits. The first panel of Figure~\ref{f:rank}, derived from demand
estimates from Ellison and Ellison (2009a), shows how a retailer's
daily sales vary with rank. The bulk of sales go to the two or three
lowest-priced retailers in this market, but positive sales still
accrue to many additional retailers on the list. For later reference
in our calculation of firm profits, we will label the curve
$Q(\Rank_{it})$.  A significant source of profit in this market came
from the ``upselling'' strategy documented by Ellison and Ellison
(2009a), by which a firm attracts potential customers with low prices
for the ``base'' memory module, but then tries to induce them to
upgrade to a more expensive one.  The second panel of
Figure~\ref{f:rank} again uses results from Ellison and Ellison
(2009a) to back out an estimate of the hourly profit from this
upselling strategy as a function of the firm's rank. The shape results
from two opposing forces: at higher ranks the firm has fewer potential
customers, but the customers it does attract are advantageously
selected be more likely to upgrade. For later reference in our
calculation of profits, which will incorporate returns from the
upselling strategy, we will label the curve $U(\Rank_{it})$.

Returning to Figure~\ref{f:series}, it illustrates another feature of
the Pricewatch market that will play a large role in our analysis:
heterogeneity in retailers' pricing strategies. We have already seen
that retailers varied in the frequency of price change. They also
appear to have targeted different segments of the ranking, with the
first firm maintaining low ranks, the second allowing itself to drift
into middle ranks, and the third content with high ranks. To
accommodate this heterogeneity in the later estimation, we will allow
the parameters to vary freely across different types of firms. In
fact, looking ahead to Section~\ref{ss:Heterogeneity}, the three firms
in Figure~\ref{f:series} are representatives of the three types into
which we will ultimately classify our sample using machine-learning
methods. This analytical treatment allows us to be agnostic about the
source of the firms' strategy heterogeneity, whether differences in
firm costs, such as the cost incurred by a manager to monitor market
conditions or to compute and enter a price change, or differences in a
firm's ability to convert customers attracted by its rank position
into sales of the base product or upgraded product (i.e., differences
in the $Q(\Rank_{it})$ or $U(\Rank_{it})$ function.

To summarize the main take-aways from this brief overview of the
market, competition among retailers for Pricewatch rank, a key
competitive variable, led to frequent although far from continuous
price changes. Our interview information suggested that this pricing
inertia was due in part to managerial frictions, motiving further
reduced-form work providing clearer documentation of managerial
inertia and motivating structural estimation of a model of managerial
costs involving in monitoring the market and changing price. Visual
evidence of substantial heterogeneity in firms' pricing strategies
will motivate our accommodating this heterogeneity in the estimation
in a flexible way.

%%%%%%%%%%%%%%%%%%%%%%%%%%%%%%%%%%%%%%%%%%%%%%%%%%%%%%%%%%%%%%%%%%%%%%%

\Section{Data}
\label{s:Data}

Our data come from Ellison and Ellison (2009a). The authors scraped
information on the first two pages of listings (12 listings per page
for a total of the 24 lowest-price listings) from the Pricewatch
website for several computer components. We focus on the category of
128MB PC100 memory modules because it is the most active and highest
volume of the categories collected and because of the homogeneity of
the products in this category. The authors recorded the name of the
firm (as well as other information about the firm such as its
location), the name of the specific product, and its price. By linking
names of firms and products over time, we are able to trace pricing
strategies of individual firms for individual products (taking the
conservative approach of assuming that a change in the product's name
indicates a change in product offering). The authors scraped this
continuously updated information every hour from May 2000 to May 2001
(with a few interruptions).

Ellison and Ellison (2009a, 2009b) supplemented the Pricewatch data
with proprietary data from a retailer who sold through Pricewatch.
This retailer provided information on its quantity sold and wholesale
acquisition cost. We take this cost to be common across retailers
justified by the fact that the typical retailer carried little
inventory, did not have long-term contracts with suppliers, and had
access to a similar set of wholesalers.

A large number of firms made brief appearances on the Pricewatch
lists.  Since we are interested in the dynamics of firms' pricing
patterns, we study firms that were present for at least 1,000 hours
during the year (approximately one-eighth of our sample period) and
changed price while staying on the list at least once.  For a small
number of firms who had multiple products on the first two pages of
Pricewatch simultaneously during some periods, we excluded their
observations during those periods.  We were left with 43 firms
appearing at some point during the year, at most 24 present on the
first two pages of Pricewatch at any particular moment. Although the
excluded retailers do not constitute observations, we do use them to
compute relevant state variables for rivals including rank, density of
neighboring firms, etc.

Based on these data, we created a number of variables to describe
factors that might be important to firms' decisions about timing and
magnitude of price changes. Figure~\ref{f:series} demonstrated the
importance of the rank of that firm on the Pricewatch list for firm
outcomes. We also included margin, length of time since its last price
change, number of times a firm has been ``bumped'' (i.e., had its rank
changed involuntarily) since its last price change, and so forth.
Table~\ref{t:dstat} provides a description of these variables and
summary statistics.

Most of the variables can be understood from the definitions in the
table, but a few require additional explanation.  \Placement\ measures
where a firm is between the next lower- and next higher-priced firms
in price space. For example, if three consecutive firms were charging
\$85, \$86, and \$88, the value for \Placement\ for the middle firm
would be $0.33 = (86 - 85)/(88 - 85)$.  \Density\ is a measure of the
crowding of firms in the price space around a particular firm. It is
defined as the difference between the price of the next higher-priced
firm minus the price of the firm three spaces below divided by 4. For
example, if five consecutive firms charged \$84, \$84, \$85, \$86, and
\$88, the value for \Density\ for the firm charging \$86 would be $1 =
(88 - 84)/4$.  \QuantityBump\ reflects relative changes in a
retailer's order flow caused by being bumped from its rank. It is
calculated using the $Q(\Rank_{it})$ function in
Figure~\ref{f:series}, in particular proportional to $\ln
(Q(\Rank_{it}) / Q(\Rank_{it'}))$, where $t$ is the current period and
$t'$ is the period in which the retailer last changed
price. \CostTrend\ and \CostVol\ are computed by regressing the
previous two weeks of costs on a time trend and using the estimated
coefficient as a measure of the trend and the square root of the
estimated error variance as a measure of the volatility.  The
definitions of the remaining variables are self-explanatory.

Turning to the descriptive statistics, we see that the average price
for a memory module in our sample was \$69, with a considerable
standard deviation of 35.1. Most of this variation is over time, with
prices typically above \$100 at the beginning of the period and down
in the \$20s by the end, mirroring a large decline in the wholesale
cost of these modules. The mean spell between price changes was 117.55
hours---about five days---similar to what we saw for the first and
third of the representative firms in Figure~\ref{f:series}. Wholesale
cost showed a strong downward trend over our sample period, falling an
average \$0.19 per day, but was quite volatile.

%%%%%%%%%%%%%%%%%%%%%%%%%%%%%%%%%%%%%%%%%%%%%%%%%%%%%%%%%%%%%%%%%%%%%%%

\Section{Reduced-Form Evidence}
\label{s:Reduced}

Before diving into the structural model, we pause in this section to
report several results from a reduced-form analysis of the
data. Besides motivating the setup of the structural model, the
results contribute independent insights into price-changing behavior,
which have the appeal of being non-technical and fairly
assumption-free.

%----------------------------------------------------------------------

\SubSection{Firm Heterogeneity}
\label{ss:Heterogeneity}

To accommodate the strategy heterogeneity illustrated by the
representative firms in Figure~\ref{f:series} in our subsequent
empirical analysis, we will adopt a compromise between combining all
firms in one sample and performing separate firm-by-firm analysis.
Separate firm-by-firm analysis sacrifices power, perhaps
unnecessarily; results in a profusion of parameters that are hard to
digest; and, most importantly, selects a non-random set of firms,
eliminating almost all of the less active firms due to too few
observations.  To balance these three concerns with a desire to allow
heterogeneity in our model, we will classify firms into a small number
of strategic types and allowed the model parameters to differ freely
across the types.

To partition the firms into types, we employed a popular
machine-learning technique, cluster analysis.\footnote{See Romesburg
  (2004) for a textbook treatment. To be clear, the cluster analysis
  referred to here is not the same as ``clustering the standard
  errors,'' which is the familiar way of adjusting standard errors for
  correlation among related observations (which as indicated in the
  table notes we do throughout, clustering by firm).}  The first step
is to select a set of dimensions along which the firms could be
differentiated. We chose seven variables that we judged were
instruments under the firms' control rather than outcomes depending on
external factors.\footnote{The variables include the firm's target
  \Rank\ and \Placement, the firm's mean values of \NumBump,
  \SinceChange, and \FirstPage, and the firm's variance of
  \SinceChange. We also included the fraction of time the firm was
  present in our sample. Firm $i$'s target value of a given variable
  is the mean of that variable computed for the subset of periods that
  immediately follow a price change by $i$.} Next, the variables are
standardized so that each has a standard deviation of 1, preventing
the variable with the largest variance from dominating the
assignment. Starting with every firm its own cluster, the algorithm
proceeds by identifying which clusters are most similar, measured by
the sum of Euclidean distances between all firms in the two clusters,
and iteratively combines them.\footnote{The method of starting with
  each item in a separate cluster and combining them until the target
  is reached is called the agglomeration method. The use of Euclidean
  distance (the sum of squared differences in the standardized
  variables) to measure difference and the criterion of combining
  clusters that have the smallest sum of squared differences is called
  Ward's method. These are the standard options in Stata 14's cluster
  command, which we used to perform the cluster analysis.}  We
iterated until three clusters were left, which we thought achieved a
balance between spanning most of the important firm heterogeneity and
still obtaining enough of each type enough to analyze empirically with
some confidence. We will allow estimates of most structural parameters
to freely vary across the three clusters, henceforth referred to as
types $\tau = 1,2,3$.

Table~\ref{t:hetero} provides variable means by each of the three firm
types.  The 22 firms of type~1 generally occupy the lower price ranks
and change prices relatively frequently.  The eight firms of type~2
occupy the middle price ranks and tend to change prices infrequently.
The 13 firms of type~3 generally charge the highest prices but are
more active in changing prices than type~2. This characterization
echoes what we saw in Figure~\ref{f:series} for the three firms
representing each type.  The means for \Margin\ show that type~1 firms
earn the lowest margins (at least captured by this measure), followed
by type~2, followed by type~3. The means of \Rank\ show the same
pattern, with type~1 occupying the lowest ranks, followed by type~2,
followed by type~3. The means of \SinceChange\ show that type~1 and
type~3 change price more than twice as often as type~2.

%----------------------------------------------------------------------

\SubSection{Distribution of Price Changes}
\label{ss:Distribution}

Figure~\ref{f:distributions} provides more refined evidence on the
distribution of the size and spells of price changes by firm
type. Panel~A provides histograms of the size of price changes. The
bar for zero change has been omitted for readability since firms did
not change price in the vast majority of hours. Note that the height
of the other bars are unconditional masses, i.e., not conditioned on a
price change. The distributions are roughly unimodal for the three
types, the mass falling off for larger magnitude price changes. The
mode for all types is at a price reduction of \$1. While there is mass
for price increases as well as reductions, more mass is on the
reduction side, consistent with the downward trend in wholesale costs
throughout our sample. A characteristic feature of type~2 firms is
that they change price only infrequently, which shows up in the
histograms as relatively little overall mass in the histogram compared
to the other types. The ratio of tail mass to mode mass is higher for
type~2 firms than others, suggesting that their infrequent price
changes tend to be larger in magnitude than other types'.

Panel~A can be related to the theoretical and empirical literatures
from macroeconomics on price stickiness. Our histograms are roughly
unimodal, inconsistent with theoretical models such as Golosov and
Lucas (2007) that, under the assumption of cheap monitoring and
expensive price changing, predict bimodal distributions, with most of
the mass on large price changes in the tails than on small price
changes in the center of the distribution. We will actually provide a
microfoundation for the unimodal price distributions shown here when
we estimate managerial costs.  In particular, we estimate a high cost
of monitoring and low cost of price changing. The empirical macro
literature has found unimodal distributions in other settings such as
Figure~2 in Midrigan (2011) and Figure~II in Klenow and Kryvtsov
(2008). The magnitudes of our firms' price changes are smaller than in
Klenow and Kryvtsov (2008), who report over 35\% of price changes
exceeding 10\% of the initial price in absolute value, while only 7\%
our our price changes are this large.

Panel~B plots a kernel density estimate of the distribution of spells
between price changes. The distribution looks similar for types~1
and~3. These frequent price changers necessarily have shorter
spells. Most of the mass is concentrated in spells shorter than 200
hours. The mass for type~2 firms spreads out more uniformly over
longer spells, even as high as 600 hours. 

The distribution of spells that we report has a shape similar to those
previously documented, but with a compressed scale. For instance, the
spell distributions for our type~1 and type~3 firms look quite similar
to Figure~IV in Klenow and Kryvtsov (2008) and Figure~VIII in Nakamura
and Steinsson (2008), but instead of a time scale stretching to 12 or
18 months respectively, ours covers just one month. The difference
could be due in part to the nature of the data: much of the previous
literature used monthly price data, missing more frequent price
changes that our hourly data would pick up. However, the analysis of
survey data in Blinder {\it et al.} (1998) finds that a large majority
of managers change the prices of their most important products no more
than twice in a year. The managers of our firms are simply more active
than those in these other studies, perhaps due to volatile wholesale
costs or low managerial costs.

Panel~C plots the absolute value of the price change as a
nonparametric function of spell using Cleveland's (1979) locally
weighted regression smoothing (LOWESS). The graph for type~1 firms
shows an upward slope, with larger changes made after longer
spells. The type~2 graph is also initially upward sloped. There is a
downturn for longer spells, but this non-monotonicity is estimated
from too few observations to be conclusive.  The type~3 graph shows a
flat relationship between size and spell. The graph for type~2 is
higher than the other types', indicating that the magnitude of
type~2's price changes tend to be larger even conditional on spell.
Although there is not a single pattern emerging here that can be used
to make a clean comparison, our results at least are not at odds with
what the previous literature has found for the relationship between
spell length and size of price change, such as Figure~VIII in Klenow
and Kryvtsov (2008).

%----------------------------------------------------------------------

\SubSection{Managerial Inertia}
\label{ss:Inertia}

We have seen that price changes, while frequent, were far from
continuous, on the order of once per week rather than each hour. This
pricing inertia could be due in principle to many factors. The integer
constraint on prices could combine with only slowly moving market
forces to produce the infrequent price changes. Perhaps firms are
reluctant to engender a rival response to a price cut. Figure~\ref{f:time}
provides evidence that the inertia is due at least in part to costs of
managerial activity.

The figure plots the residual probability of a price change (after
partialling out other covariates as controls) during each hour
measured in Eastern Time, estimated separately for retailers operating
on the East and West Coasts. Retailers supply a national market via
Pricewatch, so if there were no managerial costs, presumably the
timing of activity would be similar on the two coasts, responding to
the same national demand factors. In fact the probability functions
peak at different points, at 11 a.m.\ for East Coast and 8 p.m.\ for
West Coast retailers.

Interestingly, the peak for the West Coast retailers is not
simply shifted by the three-hour difference between Eastern Time and
the local time for West Coast retailers, as might be expected if the
retailers on the two coasts served separate markets but faced the same
pattern of managerial costs during the day. Instead, the peak on the
West Coast is shifted by ten hours, to 8 p.m. Eastern Time, which is 5
p.m. in their local time. One explanation, supported by the interview
subject, is that the market has already been operating for a few hours
by the time the West Coast managers arrive at work, so they are
well-advised to set prices in the evening before they leave.  The
evening might also be a less busy time for them since orders might
have started falling off at least from customers in the East. In
contrast, the East Coast manager would arrive to a very slow order
flow at 8 a.m.\ Eastern Time and have the leisure to adjust prices at
that point before orders picked up for the day. In the absence of
managerial costs, it is difficult to rationalize why retailers on the
two coasts would pick the opposite ends of the day to do most of their
price changes.

%%%%%%%%%%%%%%%%%%%%%%%%%%%%%%%%%%%%%%%%%%%%%%%%%%%%%%%%%%%%%%%%%%%%%%%

\Section{Structural Model}
\label{s:Model}

This section presents the model that will be the basis for our
structural estimation. We begin in the next subsection by discussing
the key object in dynamic structural estimation, the firm's value
function. The subsections after that discuss the components that go
into computation of the value function.

%----------------------------------------------------------------------

\SubSection{Value Function}
\label{ss:Value}

Firm $i$ participates in the market each period from the current one,
$t$, until it exits at time $T_i$. Its objective function is the
present discounted value of the stream of profits given by the value
function
\begin{equation}
\label{disprofit}
V_i(s_t; \sigma, \theta) = E \left[ \sum_{k=t}^{T_i} \delta^k \pi_{ik}
  \right],
\end{equation}
where $s_t$ represents the state of the game in period $t$, $\sigma$
is the vector over $i$ of firms' strategies $\sigma_i$, $\delta$ is
the discount factor,
\begin{equation}
\label{pi}
\pi_{it} = \pi(s_t, \sigma(s_t), \theta)
\end{equation}
is the static profit function, reflecting the payoff from one period
(one hour in our empirical setting) of play, and $\theta$ is a vector
of parameters. The expectation is taken over the distribution of all
possible game play and evolution of private shocks starting from
$s_t$.

The focus of this study is on obtaining structural estimates of
$\theta$, which will include measures of managerial costs, upselling
profits, and other variables of central interest. In essence, we will
compare firm $i$'s value function when it plays equilibrium strategy
$\sigma_i$ to that when it plays some deviation $\tilde{\sigma}_i$,
maintaining rivals' play as specified by $\sigma$.  The estimated
$\theta$ will be those values minimizing violations of the dominance
of $\sigma_i$ over $\tilde{\sigma}_i$. As in BBL, we are not required
to solve for the equilibrium.  Instead, we can observe the equilibrium
from the data. Still, we need to compute the value function in and out
of equilibrium, which in turn requires three components:
(a)~specification of the profit function $\pi_{it}$; (b)~an estimate
of firm $i$'s policy function, $\hat{\sigma_i}(s_t)$, which, following
BBL, we will use in place of the equilibrium strategy $\sigma_i(s_t)$;
and (c)~treatment of the expectations operator. The remainder of this
section will be devoted to specifying the profit function and policy
function. Expectations will be computed by averaging the present
discounted value of the profit stream from many simulated runs of the
market, as described in more detail in Section~\ref{s:Structural} on
the structural estimation.

For tractability and to better suit our empirical setting, our model
will depart in several ways from the standard BBL framework. We touch
on the departures here but discuss them in more detail below in the
relevant sections. First, either because of limited information,
cognition, or memory, the manager is assumed to only be able to
consider a restricted set $\hat{S}$ of the overall state space. For
example, rather than keeping track of the whole price history for each
rival, manager $i$ may view his current rank as an adequate summary
statistic of the combined effect of these actions. Second, the manager
is assumed to follow a rule-of-thumb policy that is a function of the
restricted state space $\hat{S}$. He does optimize, not over the price
each moment, but over the long-run choice of this policy, maximizing
the present discounted value of profits over the set of admissible
policies $PF$. Given the overwhelming complexity of the
period-by-period optimization problem involving the full state space,
it is unlikely that managers were fully neoclassical---all the more in
our setting in which we have qualitatively documented substantial
managerial frictions, which we aim to quantify. Our approach has the
advantage of being able to accommodate behavioral managers. The
disadvantage of model is that it will be misspecified if managers are
more neoclassical or behavioral in a different way than we allow. It
will thus be essential for us to specify a class of policy functions
that can closely match observed firm behavior. We will devote careful
attention to this specification and to gauging its resulting goodness
of fit.

%----------------------------------------------------------------------

\SubSection{Profit Function}
\label{ss:Profit}

Our detailed specification of the profit function $\pi_{it}$ draws on
our rich information about the business strategies of the firms
operating in this market and the costs they face, as well as
institutional details and estimates from Ellison and Ellison (2009a):
\begin{equation}
\label{profit}
\pi_{it} = \Base_{it} + \Upsell_{it} - \mu_{\tau} \Monitor_{it} -
\chi_{\tau} \Change_{it},
\end{equation}
where
\begin{eqnarray}
\Base_{it} & = & Q(\Rank_{it}) (\Price_{it} - \Cost_t) \label{base} \\
\Upsell_{it} & = & U(\Rank_{it}). \label{upsell}
\end{eqnarray}
The first term $\Base_{it}$ accounts for the profits from the sale of
the base version of the memory modules, equaling the quantity sold
times the margin per unit. The quantity sold is the function
$Q(\Rank_{it})$ from Figure~\ref{f:rank}A.  The second term
$\Upsell_{it}$ accounts for the upselling strategy discussed in
Section~\ref{s:Setting}, whereby the firm attracts potential customers
with low prices for the base product but then induces some of them to
upgrade to more expensive versions of the memory module. It is given
by the function $U(\Rank_{it})$ from
Figure~\ref{f:rank}B.\footnote{\label{fn:25}We normalize the profit
  function for a firm that moves off of the first two Pricewatch pages
  to \$4.70, our projection of a firm's hourly profit at rank 25 based
  on Ellison and Ellison's (2009a) parametric estimates for ranks
  1--24. The structural estimates are robust to the choice of this
  normalization.}

The final two terms in \eqref{profit} reflect the two types of
managerial costs that we seek to estimate, both of which vary by firm
type $\tau$. The coefficient $\mu_{\tau}$ on the indicator
$\Monitor_{it}$ for whether firm $i$ monitors in period $t$ will
provide an estimate of the cost of monitoring.  The last term involves
$\Change_{it}$, which is our label for the indicator $1\{\Delta_{it}
\neq 0\}$ for whether firm $i$ changes price in period $t$, where we
define $\Delta_{it} = \Price_{it} - \Price_{i,t-1}$. The coefficient
$\chi_{\tau}$ on this indicator will provide an estimate of the cost
of price change over and above any monitoring cost. We do not observe
monitoring, but, just as we will generate simulated values for
$\Rank_{it}$, $\Price_{it}$, and other variables when we compute
expectations of the value function, we can simulate $\Monitor_{it}$
using the model of monitoring behavior embedded in the policy function
discussed in the next subsection.

The specification of managerial costs $\mu_{\tau}$ and $\chi_{\tau}$
as fixed parameters represents a departure from BBL's framework, which
specifies a distribution over structural parameters with a mean and
variance to be estimated. Our approach of estimating a single fixed
parameter eases the computational burden, important given we allow the
parameters to differ across the three types $\tau$. We also have
several conceptual reasons for our approach. The policy function
specified in the next subsection already incorporates arrival of
opportunities for managerial actions based on both stochastic as well
as deterministic factors. Our estimates of the structural parameters
can be thought of as the marginal cost of taking the relevant
managerial action in opportune periods dictated by the policy
function. Implicit in this formulation is the presumption that
managerial actions are not taken in other periods in part because of
high, perhaps prohibitive, draws for random managerial costs. To
explain the relative rarity of price changes observed on a hourly
basis, an estimated distribution of managerial costs would have to be
skewed toward large values compared to our estimate for opportune
periods. Estimating the distribution of managerial costs would require
specifying the functional form for the distribution and the
hour-to-hour correlation structure, which could be quite complex. Our
policy function captures the time-series properties of opportunities
for managerial action simply, by including covariates such as
\SinceChange, \Night, and \Weekend, among others.

%----------------------------------------------------------------------

\SubSection{Policy Function}
\label{ss:Policy}

The next piece needed for structural estimation is the policy
functions, or models of firms' strategies. An estimate of this model
provides the $\hat{\sigma_i}(s_t)$ that will be substituted for
equilibrium strategies $\sigma_i(s_t)$ to compute the value function
\eqref{disprofit} in the simulations. The model reflects the reality
that price changes in this market are infrequent relative to our data
frequency. To a first order, the best prediction of the firm's price
next hour is its current price. We will thus focus on modeling the
timing and size of price change episodes, with the firm maintaining
the price outside of these episodes. Consistent with our belief that
firms are not engaging in complicated calculations of optimal policy
based on hundreds of state variables each hour, we want the model to
be simple, streamlined, and reflective of just the variables that
firms are likely to be able to monitor and process.  Also, we want the
model to be a predictive empirical description of what firms actually
do.

We thus model price changes as coming from a two-step process. The
manager knows some components of the market state vector at all times,
information he receives essentially ``for free.''  Based on these
state variables, the manager's first step is to decide whether to
attend to the market to gather information or perform calculations
needed for a pricing decision. We will call this behavior
``monitoring'' and denote the decision to do so with the indicator
function $\Monitor_{it}$.  In our setting, we think of monitoring as
computing quantities like percentage markups and visiting the
Pricewatch website to gather the relevant information, involving an
opportunity cost of cognition and time. Through monitoring, the
manager gains additional information on state variables, including
current rank and the distribution of competitors' prices, and computes
the new desired price. If the new desired price is different from the
current price and the costs of changing it are justified, then he
enters it in the Pricewatch form, and changes it on his own website,
again involving costs in terms of cognition and time.

The two-step process can account for periods of excess inertia, during
which the manager keeps price constant even though market conditions
would warrant a price change. Inertia can come from three
sources. First, the manager may not be aware of the changed market
conditions because he did not monitor. Second, the benefit from making
a desired price change, especially a small price change, may not
justify the managerial cost of entering it. Third, if the desired
price change is smaller than a whole dollar unit in which Pricewatch
prices are denominated, price may stay constant. The two-step process
is also consistent with anecdotal evidence from our interview subject
and broader survey evidence (see Blinder et al., 1998) that managers
often monitor market conditions including rival prices without
changing their own price.

We specify the manager's latent desire to monitor, $\Monitor^*_{it}$,
as
\begin{equation}
\label{monitoring}
\Monitor_{it}^* = X_{it} \alpha_{\tau} + e_{it},
\end{equation}
where $X_{it}$ is a vector of explanatory variables, $\alpha_{\tau}$
is a vector of coefficients to be estimated, which are allowed to
differ across firm types $\tau$, and $e_{it}$ is an error term. If
$\Monitor^*_{it} \geq 0$, then the firm monitors; {\it i.e.},
$\Monitor_{it} = 1$. Otherwise, if $\Monitor^*_{it} < 0$, then the
firm does not monitor; {\it i.e.}, $\Monitor_{it} = 0$.

In essence, equation \eqref{monitoring} embodies the manager's
forecast of the costs and benefits of monitoring, so, perforce, the
explanatory variables can only include state variables known by the
manager before monitoring. In addition, the variables must be
important shifters of either the cost or benefit of monitoring. We
specify the following parsimonious list:
\begin{multline}
\label{X}
X_{it} = \biggl(\Night_t, \, \Weekend_t, \, \CostVol_t, \,
\CostTrend_t^{\,+}, \, \vert \CostTrend_t^{\,-} \vert \\ 
\QuantityBump_{it}^+, \, \vert \QuantityBump_{it}^- \vert, \, \ln
\SinceChange_{it}, \, (\ln \SinceChange_{it})^2 \biggr).
\end{multline}
We assume the manager is automatically aware of the time and day. The
variables $\Night_{t}$ and $\Weekend_{t}$ are included to reflect that
the cost of monitoring varies in a predictable way over a week: the
cost of monitoring at 2am on Sunday morning might be high if that's
when a manager typically sleeps, and the benefit might be low because
few sales would be made around that time anyhow.  The manager is also
assumed to be aware of the day's wholesale cost---recall that he
receives emails every day from the wholesalers---and can glean
volatility and trends from the pattern of costs over the past couple
of weeks. Presumably the gains to monitoring are greater the more
conditions including costs are fluctuating. We include $\CostVol_{t}$
to capture the magnitude of recent unpredictable fluctuations and
$\CostTrend_t$ to capture recent trends. Both rapidly rising and
rapidly falling costs would lead the manager to monitor more. To allow
asymmetry\footnote{Some evidence suggests that prices could be
  stickier in one direction versus the other.  For instance,
  Borenstein, Cameron, and Gilbert (1997) identify an asymmetry in the
  response of wholesale gasoline prices to cost increases versus
  decreases.} in the concern for rising or falling costs, a rising
cost trend, $\CostTrend_t^{\,+} = \CostTrend_t \times 1\{\CostTrend_t
> 0\}$, enters \eqref{monitoring} separately from a falling cost
trend, $\CostTrend_t^{\,-} = |\CostTrend_t| \times 1\{\CostTrend_t <
0\}$. The latter variable appears in absolute value so that we would
anticipate the two variables' coefficients to have the same sign if
not magnitude.

We assume that the manager is roughly aware of changes in his order
flow resulting from being bumped in the ranks and will be more likely
to monitor if there has been a large change, whether an increase or
decrease. Thus \eqref{monitoring} includes
$\QuantityBump_{it}$. Recall that this variable is computed by
translating the current rank and rank at the previous price change
into a quantity change using the function in
Figure~\ref{f:rank}A. This predicted change in order flow is a proxy
for the quantity signal the manager observes. The proxy diverges from
the signal because the firm's actual sales depend on random market
fluctuations on top of any predictable effect of a rank change. The
proxy also diverges from the signal because the manager may only be
vaguely aware of actual sales in any given hour. Again, to allow for
asymmetries, increases in order flow, $\QuantityBump_{it}^+$, enter
separately from decreases, $\QuantityBump_{it}^-$.

The last set of variables, functions of $\SinceChange_{it}$, enter in
a flexible, nonlinear way, allowing for various patterns of managerial
attention, including monitoring the market at regular intervals as
well as periods of intense monitoring, in which several price changes
may follow in succession, followed by periods of inattention during
which price stays constant independent of market conditions.
Including this variable can help us tease out time dependence from
state dependence in price monitoring behavior.  

Conditional on monitoring, the manager may decide to change price
based on the information acquired. Let $\Delta_{it}^* = \Price_{it}^*
- \Price_{i,t-1}$ denote the size of the price change the manager
would desire if price were a continuous variable and the menu cost
$\chi_{\tau}$ of changing price were ignored just for period
$t$. Assume this latent variable is given by
\begin{equation}
\label{latentpricechange}
\Delta_{it}^* = \begin{cases}
 Z_{it} \beta_{\tau} + u_{it} & \text{if $\Monitor_{it} = 1$} \\
 0 & \text{if $\Monitor_{it} = 0$,}
\end{cases}
\end{equation}
where $Z_{it}$ is a vector of explanatory variables, $\beta_{\tau}$ is
a vector of coefficients to be estimated, which again are allowed to
differ across firm types $\tau$, and $u_{it}$ is an error term. The
actual price change $\Delta_{it} = \Price_{it} - \Price_{i,t-1}$ may
diverge from the latent price change $\Delta_{it}^*$ for two
reasons. First, rather than being continuous, prices were denominated
in whole dollars on Pricewatch. Second, because changing price is not
in fact costless, the manager must weigh the benefits of changing
price if cost were no object reflected by $\Delta_{it}^*$ against the
menu cost $\chi_{\tau}$. Assuming the manager's willingness to pay to
change price is an increasing function of the size of the desired
price change, we can relate the observed price change $\Delta_{it}$ to
the actual $\Delta_{it}^*$ by specifying cut points along the real
line
\begin{equation}
\label{C}
\cdots < \cut{-5} < \cut{-4} < \cut{-3} < \cut{-2} < \cut{-1} <
\cut{0} < \cut{1} < \cut{2} < \cut{3} < \cut{4} < \cut{5} < \cdots .
\end{equation}
Then
\begin{equation}
\label{pricechange}
\Delta_{it} = \begin{cases}
k & \text{if $\Delta_{it}^* \in \bigl( \cut{k}, \cut{k+1} \bigr)$,} \\[1ex]
-k & \text{if $\Delta_{it}^* \in \bigl( \cut{-(k+1)}, \cut{-k} \bigr)$.}
\end{cases}
\end{equation}
Thus, for example, an observed price increase of \$1 corresponds to a
latent price change satisfying $\Delta_{it}^* \in \bigl( \cut{1},
\cut{2} \bigr)$. If the manager monitored but did not change price,
then $\Delta_{it}^* \in \bigl( \cut{-1}, \cut{1} \bigr)$. To deal with
the fact that many elements of the manager's decision to change
price---the function mapping the size of his desired price change into
his willingness to pay to change price, the distribution of the error
term $u_{it}$, of course the menu cost $\chi_{\tau}$, which is yet to
be estimated---we adopt a flexible specification for the $\cut{k}$,
allowing them to be free parameters estimated in a similar way as the
cut points in an ordered probit, and allowing them to differ across
firm types $\tau$.

The explanatory variables can include state variables the manager
learns as a result of monitoring in addition to those known before. We
specify the following parsimonious set:
\begin{multline}
\label{Z}
Z_{it} = \biggl(\CostTrend_t, \, \CostChange_{it}, \, \Margin_{it}, \, \NumBump_{it}, \\
 \Density_{it} \times \NumBump_{it}, \, \Placement_{it}, \, \Rank_{it}, \, \RankOne_{it} \biggr).
\end{multline}
The higher are forecasted costs, $\CostTrend_t$, and the more costs
have risen since firm $i$ last changed its price, $\CostChange_{it}$,
the higher the firm's desired price. $\Margin_{it}$ may also factor
into price changes, a firm with low margins being more likely to
increase price and one whose margins are already high less likely to
further increase price and more likely to lower.

The remaining variables in $Z_{it}$ are the sort of state variables
revealed by monitoring. After visiting the Pricewatch website, the
firm learns its current rank and thus the number of ranks it was
bumped since the last price change. The firm can use this information
to return itself to its desired rank, so decreasing price if it was
bumped up in the ranks (gauged by a positive $\NumBump_{it}$) and
increasing price if it was bumped down. A low-ranked firm may have
less of an incentive to cut price to increase it sales; this effect
may be particularly strong for a firm occupying rank~1: further price
reductions will only result in a small increase in sales because most
of the demand elasticity is with respect to rank, which the firm
cannot improve beyond 1.  We, therefore, include an indicator,
$\RankOne_{it}$. We include $\Density_{it}$ because the presence of a
thicket of close competitors may affect pricing incentives. For
example, in a dense price space, firms may have less incentive to
increase price because this will result in a more severe rank
change. Specifically, we include $\Density_{it}$ interacted with
$\NumBump_{it}$ because density will likely matter more for firms that
have a need to change price as proxied by a bump from the previous
preferred ranking. A firm's $\Placement_{it}$ between its nearest
rivals will also affect its desired price in potentially complex ways.

The list of explanatory variables in $Z_{it}$ deliberately excludes
some of the variables that appear in the monitoring equation. For
example, $\Night_t$ is included in the monitoring equation to reflect
the fact that checking the Pricewatch website at 2 a.m. would
typically be more costly than during the workday. However $\Night_t$
should have little effect on the desired price change conditional on
monitoring because, conditional on already being on the website,
changing price is no more difficult at 2 a.m. than 10 a.m. The same
logic applies to $\Weekend_t$. While the time since price was last
changed may affect the desire to monitor---one possibility is that
with more time for market conditions to change, more information is to
be gained---conditional on the information gained through monitoring
such as $\Rank_{it}$ and $\NumBump_{it}$, $\SinceChange_{it}$ has no
obvious role in price setting and thus is excluded from $Z_{it}$.

While it is easy to justify the inclusion of the explanatory variables
in the respective functions, it may be harder to assert that this
strategy model is sufficient to describe firms' behavior. Firms could
have executed much more complex strategies, including nonlinear
functions of the included variables, interactions of them, and
additional variables such as the characteristics of neighboring firms
in the Pricewatch ranking. More generally, one might worry whether a
reduced-form policy function could ever capture the intricacies
involved in full-blown expected value maximization each period.  We
argue that our approach still has some appeal in our empirical
setting. First, our approach is computationally much less burdensome
than a fully rational model. Second, it is reasonable to suppose
managers used fairly simple rules of thumb to make the high-frequency
decisions to monitor and change price rather than continually
evaluating expected value functions based on scores of state variables
each period. Such calculations would overwhelm the direct cost of
checking websites and calculating and typing in new prices involved in
the managerial actions.  Third, as discussed in
Section~\ref{ss:Policy_Fit}, we have encouraging results from a
goodness of fit test to determine the match between simulations from
our estimated strategies and observations in the data. 

XXXXXXXXXXXXXXXXX HONGKAI, SEEMS LIKE WE NEED TO BUILD ON THIS TO EXPLAIN IT IF WE MENTION IT XXXXXXX
Finally, we
applied a machine-learning method (gradient-boosted machine) to
generate a policy function without imposing functional-form
restrictions on the included variables. Goodness of fit was not
significantly improved by the machine-learning method relative to our
reported policy function. XXX WHERE DOES THIS COME FROM? MORE ON
THIS. XXX

%%%%%%%%%%%%%%%%%%%%%%%%%%%%%%%%%%%%%%%%%%%%%%%%%%%%%%%%%%%%%%%%%%%%%%%

\Section{Estimation of the Policy Function}
\label{s:Policy}

%----------------------------------------------------------------------

\SubSection{Estimation Details}
\label{ss:Policy_Details}

Our model for the policy function embodied in equations
\eqref{monitoring}--\eqref{Z} is known in the econometrics literature
as a zero-inflated ordered probit (ZIOP). The term ``zero-inflated''
refers to the fact that there are more zeros---in our setting periods
without a price change---than would be expected under a standard
ordered probit. The overabundance of zeros in our setting results from
the manager's monitoring less than continuously because of the cost of
monitoring. Thus the monitoring stage determines the degree of
zero-inflation.  Harris and Zhao (2007) demonstrate in Monte Carlo
experiments that the maximum likelihood estimator of the ZIOP performs
well in finite samples. While we adopted the ZIOP specification and
chose the variables appearing in the each stage to best reflect our
understanding of the empirical setting informed by manager interviews,
formal Vuong (1989) tests of non-nested models reported below support
the inclusion of a monitoring stage to the price-change equation and
the support our choice of variables to include in the different
stages.

We estimate equations \eqref{monitoring}--\eqref{Z} jointly
using maximum likelihood taking the errors $e_{it}$ and $u_{it}$ to be
independent standard normal random variables. 

Without the monitoring
stage, the price-change stage of the model would simply be an
ordered probit, where various intervals would correspond to various
discrete price changes. With the monitoring stage, the likelihood of
observations in which firm $i$ does not change price at time $t$ is
\begin{equation}
\label{LF0}
L(\Delta_{it}=0) = \underbrace{1 - \Phi(X_{it}
  \alpha_{\tau})}_{\Pr(\scriptsize\Monitor_{it} = 0)} +
\underbrace{\Phi(X_{it} \alpha_{\tau})}_{\Pr(\scriptsize\Monitor_{it}
  = 1)} \underbrace{\left[ \Phi \bigl( \cut{1} - Z_{it} \beta_{\tau}
    \bigr) - \Phi \bigl( \cut{-1} - Z_{it} \beta_{\tau} \bigr)
    \right]}_{\Pr \bigl( \scriptsize\Delta_{it}^* \in \bigl(\cut{-1},
  \cut{1} \bigr) \bigr)},
\end{equation}
where $\Phi$ is the standard normal distribution function. The
likelihood of, for example, a $k$ dollar price increase is
\begin{equation}
\label{LFk}
L(\Delta_{it} = k) = \underbrace{\Phi(X_{it}
  \alpha_{\tau})}_{\Pr(\scriptsize\Monitor_{it} = 1)}
\underbrace{\left[ \Phi \bigl( \cut{k+1} - Z_{it} \beta_{\tau} \bigr)
    - \Phi \bigl( \cut{k} - Z_{it} \beta_{\tau} \bigr) \right]}_{\Pr
  \bigr(\scriptsize\Delta_{it}^* \in \bigl( \cut{k}, \cut{k+1} \bigr)
  \bigr)}.
\end{equation}
Adding the monitoring stage scales up the probability of no price
change and scales down the probability of any given sized price
change.

We group the few price reductions of \$5 or more together and
similarly group the few price increases of \$5 or more so that we only
need to estimate thresholds down to $\cut{-5}$ and up to $\cut{5}$.

%----------------------------------------------------------------------

\SubSection{Identification}
\label{ss:Policy_Identification}

Though the dependent variable $\Monitor_{it}$ in the monitoring
equation \eqref{monitoring} is not itself observable, the coefficients
$\alpha_{\tau}$ can still be estimated because the dependent variable
$\Delta_{it}$ in the price-change equation \eqref{pricechange} is
observable, and $\Monitor_{it}$ enters that equation through
\eqref{latentpricechange}. 

The monitoring and price-changing equations combine to produce a
single price change, but the parameters in the separate stages can
still be separately identified. Intuitively, parameters in the
price-changing equation are identified from periods with active price
changes. Fixing the covariates, a unique parameter vector will best
match the distribution of price increases and decreases. The number of
zeros may not be well matched; the mismatch identifies the parameters
in the monitoring equation. For example, an excess of zeros is
attributed to a reduced likelihood of monitoring since, had the firm
had been monitoring, these opportunities to change price would not
have been missed.

In Monte Carlo exercises, Harris and Zhao (2007) find that the ZIOP
model is well identified even without exclusion restrictions, but
exclusion restrictions can help avoid convergence problems and improve
the precision of parameter estimates. Exclusion restrictions play a
more central role in applications such as ours in which
heteroskedasticity or non-normal errors cannot be ruled out, precluded
by data limitations from estimating the error distribution
nonparametrically. The problem is similar to that arising with the
Heckman (1979) selection model, which is well known to perform poorly
in the presence of non-normal or heteroskedastic errors without
exclusion of some variables appearing in the selection equation from
the main equation.

The natural exclusion restrictions we employ bolster
identification. We include $\Night_t$ and $\Weekend_t$ in the
monitoring equation but exclude them from the price-change equation
because they relate more to the inconvenience of activity at those
times than the optimal size of a price change. We also include
$\SinceChange_{it}$ in the monitoring equation, capturing the
manager's guess of how far out of line the unadjusted price has become
over time, but excluded from the price-change equation because other
variables are sufficient statistics for the optimal price change once
the manager is able to observe them through monitoring. These excluded
variables will help identify true excess zeros (when these
instrumental variables are high, say) from a distributional mode that
is higher than expected due to non-normal or heteroskedastic errors.

%----------------------------------------------------------------------

\SubSection{Results}
\label{ss:Policy_Results}

Table~\ref{t:policy} presents estimates of $\alpha_{\tau}$,
$\beta_{\tau}$, and $\cut{k}$ by firm type $\tau$. Consider the
results for type~1 firms first, as these represent the majority of
firms, comprising the biggest sample. $\Night$ and $\Weekend$ show up
as important determinants of whether a firm monitors. This result is
not surprising but is a telling indicator of the importance of
managerial attention. Managers evidently take evenings and weekends
off and do not bother monitoring the market then.  Managers are also
more likely to monitor if wholesale cost is volatile, if there has
either been a sharp trend in wholesale cost or a sharp change in order
flow recently, or if there has been a long spell since the last price
change.  These estimates are consistent with intuition and are quite
precisely estimated for type~1 firms.  Conditional on monitoring, a
price increase for a type~1 firm is associated with a wholesale cost
increase, a firm being bumped down, rank being low, rank being 1, and
there being few firms close by in price space.  Again, these results
are consistent with intuition as well as our conversations with one of
the firm managers.

A likelihood-ratio test rejects homogeneity of the policy function
across types at better than the 1\% level. While there are
quantitative differences among the results for the firm types, there
are generally not large qualitative differences, especially for the
$\alpha_\tau$ coefficients. Our framework is flexible enough to
accommodate the difference in activeness between type~2 firms and the
rest.  In the monitoring equation, the coefficients for $\ln
\SinceChange$ and $(\ln \SinceChange)^2$ of type~2 firms constitute a
quadratic function of $\ln \SinceChange$ that is decreasing when
$\SinceChange < 586$.  So a type~2 firm would only monitor when it has
accumulated enough change in $\QuantityBump$ or it has been a very
long time since the last price change.  Conditional on monitoring, a
type~2 firm requires a more significant latent size of price change to
trigger actual price changes because the size of the interval of no
price change, $\cut{1} - \cut{-1}$, is larger for $\tau = 2$ than
other types.

%----------------------------------------------------------------------

\SubSection{Goodness of Fit}
\label{ss:Policy_Fit}

This subsection assesses the goodness of fit of our estimated policy
function $\hat{\sigma}$, in particular whether it fits well enough to
be suitable for the later structural estimation. As discussed in the
next section, after substituting the specified profit function as well
as estimates of the structural parameters, firm $i$'s value function
$\widehat{V}_{i}\left(\hat{\sigma},\theta \right)$ reduces to a linear
function of just a few aggregates, in particular, the total hours firm
$i$ spends at each rank and the total times it changes its
price. Hence the goodness of fit of the estimated policy function can
be assessed by comparing aggregates simulated from the policy function
to their values in the actual data. We first provide an informal,
visual assessment using a series of figures.

To provide some technical details on how the simulations were run, we
start with a given firm type and consider all the observations at the
$it$ level for that type. For each $it$, we construct 20 simulated
forward histories lasting 720 hours. The simulations use the actual
market data for state variables where possible (e.g., for cost
histories), simulating firm behavior by substituting current state
variables as well as random draws for error terms $e_{it}$ and
$u_{it}$ into equations \eqref{monitoring} and
\eqref{latentpricechange} of the policy function. Averaging the
aggregate over all the simulations for all $it$ observations of that
firm type produces a value that can be compared to the average in the
actual data of 720-hour forward histories starting from each candidate
observation. To be precise, the aggregates are discounted rather than
simple sums, using the same annual discount factor of 0.95 used in the
value functions in the structural estimation. Over the 720-hour
horizon, discounting has a negligible effect on the results.

The first panel of Figure~\ref{f:fit} compares compares simulated time
spent at each rank (solid black curves) to actual (dashed grey curves)
for the three firm types. The actual curves have quite different
levels and shapes across firm types, yet the simulated curves are able
to fit each quite well. Panel~B compares the simulated number of price
changes from the estimated policy function (the black squares) to the
averages in the actual data (the grey circles). Again, the fit is
quite close, with the markers essentially overlapping and moving
together across the types of firm: moderate for type~1, low for
type~2, and high for type~3. The close fit is not surprising given the
policy function is a reduced form that was estimated in part to
maximize the likelihood of the observed frequency of price
changes. However, the close fit was not guaranteed. The maximum
likelihood estimation targeted size as well as number of price
changes; we see that the joint estimation does not harm the fit for
number alone. More importantly, the policy functions were estimated to
fit individual behavior; letting their interaction on the market play
out over a length of time could generate feedback causing behavior to
diverge from actual outcomes. We see in the first panel that this is
not the case.  Taken together, the results from Figure~\ref{f:fit}
suggest that the estimated policy function will provide good estimates
of the sums that are the essential inputs into the value functions in
the structural estimation.

Figure~\ref{f:states} provides a more refined assessment of goodness
of fit. The policy function should not only be able to fit the forward
history on average across states but also fit the forward history in
any state $s$ in which the firm finds itself. The figure compares
simulated to actual profit conditional on various states including
various initial ranks in the first panel and margins in the second
panel. (To save space, we just show the results for type~1 firms; the
fit for types~2 and~3 is similar.)  We focus on profit in this figure
as opposed to time spent at each rank in the previous figure because
profit is a convenient summary statistic for the distribution of times
spent at each rank, allowing us to reduce the dimensionality of the
graph. While we do not yet have all the components of profit
$\pi_{it}$ from equation \eqref{profit}---the managerial-cost
components are of course yet to be structurally estimated---we can
estimate the first two components $\Base_{it}$ and $\Upsell_{it}$
using equations \eqref{base} and \eqref{upsell}. Because monetary
profits are not observed even in the actual data, they have to be
estimated in both cases---using simulated prices in the former case
and actual prices in the latter. Because the figure displays the
results separately for each initial state $s$, each average is now
taken only over initial observations at the $it$ level that qualify
for state $s$.

Panel~A compares the simulated to actual monetary profit going forward
for 720 hours conditional on rank in the initial period. The solid
black curve for profit based on simulated prices closely matches the
dashed grey one for profit based on actual prices over the whole range
of the horizontal axis. Of course the closeness of the graphs could be
a symptom, not of good fit, but of the stability of the environment,
with initial rank correlating highly with profits at least over an
horizon as short as 720 hours. To investigate this possibility, we
have added a curve (lighter with dot markers) representing the
na\"{i}ve forecast that the firm earns the same profit in each of the
720 hours as it does in the first. This na\"{i}ve forecast ends up
overestimating profit conditional on top ranks, because it has not
taken into account the other firms' reaction to firm $i$'s top rank,
which is to undercut firm $i$. For similar reasons it ends up
underestimating profit conditional on bottom ranks. Our estimated
policy function fits dramatically better for both high and low ranks,
suggesting that our policy function likely captures this dynamic
correctly.

Panel~B similarly compares simulated, actual, and na\"{i}ve estimates
of monetary profit conditional on a type~1 firm's margin (price minus
wholesale cost in levels) in the initial state $s$. Again our
estimated policy simulation fares well, while the na\"{i}ve prediction
underestimates profit conditional on negative margins and
overestimates profit conditional on positive margins because does not
properly incorporate the firm's future price adjustments and the
cascade of rival responses. This comparison suggest that our policy
function has likely captured this dynamic correctly.

XXXXXXXXXXXXXXXXXX MY POOR ATTEMPT AT VUONG TABLE. NEEDS HONGKAI'S ATTENTION

Table~\ref{t:vuong} provides an additional assessment of the policy
function's fit using formal Vuong (1989) tests of model closeness. The
Vuong (1989) test statistic $Z$ compares the model's likelihoods after
subtracting a penalty for extra parameters. We compare our preferred
model, the zero-inflated ordered probit (ZIOP) embodied in equations
\eqref{monitoring}--\eqref{Z}, labeled model A, to a series of
alternatives, labeled B--D.

For model B, we start with our preferred specification but shift some
variables from $X_{it}$ in the monitoring equation to $Z_{it}$ in the
price-change equation; in particular we shift all the variables except
$\Night_t$ and $\Weekend_t$ from $X_{it}$ to $Z_{it}$. This comparison
will test whether the careful choice of which equation the variables
appear in matters or whether it just matters the variables are
included somewhere in the specification. For model~C, instead of
shifting the variables from $X_{it}$ to $Z_{it}$, we just omit them
from the model entirely. This comparison will test whether the
monitoring equation is mainly identified off of the timing variables
$\Night_t$ and $\Weekend_t$ or whether other included variables
contribute as well. Model~D keeps the price-change equation as
specified in the preferred Model~A but omits monitoring equation
\eqref{monitoring}. Thus, unlike models A--C, which are all ZIOP
models, model~D is a simple ordered probit (OP) model.

In cases of nested or partially overlapping models, the Vuong $Z$
statistic has a complex distribution, equal to a weighted sum of chi
squares where the weights are eigenvalues of a matrix of conditional
expectations, leading authors to provide a subjective evaluation of
the level of the $Z$ statistic rather than directly computing its
$p$-value. We see that the test statistic is enormous for the
comparison of the OP model omitting monitoring against any of the ZIOP
models. This suggests that including the monitoring equation produces
a substantially better fit. The preferred ZIOP specification has a
considerable $Z$ statistic of 3.20 and 3.92 when tested against the
alternatives that either shift or delete variables from the monitoring
equation. This provides suggestive evidence that careful placement of
the variables in the correct equations matters for fit.

XXX ARE SOME OF THESE COMPARISONS IN FACT NESTED? IF SO, SHOULD
DESIGNATE WHICH ARE AND WHICH AREN'T AND EXPRESS SIGNIFICANCE FOR
THOSE WHICH WE KNOW THE DISTRIBUTION.

%%%%%%%%%%%%%%%%%%%%%%%%%%%%%%%%%%%%%%%%%%%%%%%%%%%%%%%%%%%%%%%%%%%%%%%

\Section{Estimation of Structural Parameters}
\label{s:Structural}

%----------------------------------------------------------------------

\SubSection{Identification}
\label{ss:Structural_Identification}

For tractability and to better suit our empirical setting, our model
will depart in several ways from the standard BBL framework. We touch
on the departures here but discuss them in more detail below in the
relevant sections. First, either because of limited information,
cognition, or memory, the manager is assumed to only be able to
consider a restricted set $\hat{S}$ of the overall state space. For
example, rather than keeping track of the whole price history for each
rival, manager $i$ may view his current rank as an adequate summary
statistic of the combined effect of these actions. Second, the manager
is assumed to follow a rule-of-thumb policy that is a function of the
restricted state space $\hat{S}$. He does optimize, not over the price
each moment, but over the long-run choice of this policy, maximizing
the present discounted value of profits over the set of admissible
policies $PF$. Given the overwhelming complexity of the
period-by-period optimization problem involving the full state space,
it is unlikely that managers were fully neoclassical---all the more in
our setting in which we have qualitatively documented substantial
managerial frictions, which we aim to quantify. Our approach has the
advantage of being able to accommodate behavioral managers. The
disadvantage of model is that it will be misspecified if managers are
more neoclassical or behavioral in a different way than we allow. This
motivated our careful specification of the policy function to matching
observed firm behavior and our assessment of its goodness of fit.

We follow the broad outlines of the BBL approach to estimate our
structural parameters with a few modifications.  While BBL assume the
econometrician observes the game play of a Markov prefect Nash
equilibrium, we assume a weaker solution concept such that the
observed game play is a Nash equilibrium in which each player chooses
a rule-of-thumb policy function based on a limited set of state
variables to maximize the present discounted value of profits. The
equilibrium policy function is that estimated in the previous section.
We then calculate the simulated counterpart of the value function in
equation \eqref{disprofit}, which involves approximating the
distribution of the states considered by the firm. Assuming firms have
consistent beliefs about game play---the premise of the BBL
method---then a natural approximation of the distribution of states in
firms' consideration sets is the empirical distribution observed in
the data, denoted $\hat{S}$. This approximation is also consistent
with our policy-function estimation, which matches firms' behavior to
the same distribution of states.

In the final step, we search for structural parameters
$\theta$---which here is the vector of managerial costs $\mu_\tau$ and
$\chi_\tau$---that rationalize the estimated policy function as the
equilibrium one, i.e., the one among the class of admissible policy
functions maximizing the value function.  Let
$PF(\alpha_{\tau},\beta_{\tau},\cut{k})$ denote the admissible class
of policy functions, comprised by substituting alternative values for
the estimated parameters in the monitoring and price-changing
equations.  Our identification condition is that no deviation in
$PF(\alpha_{\tau},\beta_{\tau},\cut{k})$ can dominate the estimated
policy function $\hat{\sigma}_{i}$. Formally,
\begin{equation}
\label{identification_ours}
 E_{s\in\hat{S}} [ V_{i}(s;\hat{\sigma},\theta )] \geq E_{s\in\hat{S}}
 [V_i(s;\tilde{\sigma}_i, \hat{\sigma}_{-i}, \theta )] \quad \text{for
   all $\tilde{\sigma}_i \in PF(\alpha_{\tau},\beta_{\tau},\cut{k})$}.
\end{equation}
We estimate $\theta$ by finding values that satisfy
\eqref{identification_ours} for a large number of deviations
$\tilde{\sigma}_i$. The details of our specific estimator and our
choice of deviations are provided in the appendix.

The intuition behind how $\theta$ is identified is
straightforward. Our estimate of the equilibrium policy function
$\hat{\sigma}_{i}$ will entail some rates of monitoring and price
changing. Holding the benefits of these actions constant, their rates
should be inversely related to their costs, $\mu_\tau$ and $\chi_\tau$
respectively. For example, a high equilibrium monitoring rate entailed
by $\hat{\sigma}_{i}$ must imply a low $\mu_\tau$. If instead
$\mu_\tau$ were extremely high, the set of deviations
$PF(\alpha_{\tau},\beta_{\tau},\cut{k})$ is rich enough that one could
be found involving a lower rate of monitoring, reducing the number of
times $\mu_\tau$ is subtracted from the profit stream, which for a
given benefit of monitoring would increase the value
function. Similarly, a high equilibrium rate of price change
(conditional on the rate of monitoring) can only be consistent with a
low $\chi_\tau$. What may be difficult in practice is distinguishing
in the data an equilibrium with frequent monitoring but inert price
changing from one with infrequent monitoring but hair-trigger price
changing. But that is a difficulty with the identification of the
policy function---discussed already in
Section~\ref{ss:Policy_Identification}---not the structural
parameters. A well-identified policy function will deliver distinct
rates of monitoring and price changing that can be used to identify
$\mu_\tau$ and $\chi_\tau$. If the policy function is not well
identified, then different bootstrapped samples will lead to large
swings in the rates of monitoring and price changing, leading to large
bootstrapped standard errors on the structural parameters. Thus the
standard errors on the structural parameters will serve as a natural
check on the identification of the policy function.

To identify the structural parameters, it is crucial that the richness
of the set of deviations $PF(\alpha_{\tau},\beta_{\tau},\cut{k})$ be
exploited.  Not all deviations give useful information. For example, a
deviation generating negative margins most of the time would be
dominated by the equilibrium strategy for any managerial-cost
parameters. As discussed further in the next subsection, we will
sample from distributions of deviations that have a realistic chance
of being profitable to maximize the power of identification assumption
\eqref{identification_ours}.

The restriction of deviations to $\tilde{\sigma}_i(s_i) \in
PF(\alpha_{\tau},\beta_{\tau},\cut{k})$ is where our procedure
accommodates behavioral managers, who pursue rule-of-thumb strategies
short of fully rational strategies. A fully rational strategy would
have to dominate all conceivable deviations including, for example, a
one-time price increase of \$1 in any given hour, a deviation which is
not in $PF(\alpha_{\tau},\beta_{\tau},\cut{k})$. Given the difficulty
of solving for the fully rational strategies in our setting, we are
reluctant to examine deviations that only ``work'' (i.e., generate
inequalities in the correct direction) if firms are fully
rational. The restricted set of deviations we consider reflects the
idea that firms experiment among a simpler class of pricing rules to
discover the most profitable of them. By restricting the set of
deviations, our identification assumption is weaker than the standard
assumption in BBL, allowing for estimation of structural parameters
that is robust to certain forms of behavioral pricing. The only
potential pitfall for structural identification would be if the
deviation set were not rich enough to span possible combinations of
monitoring and price-changing rates. Our model avoids this pitfall,
however: fixing all the other parameters, varying the constant terms
in $\alpha_\tau$ and $\beta_\tau$ over $(-\infty, \infty)^2$ can
produce any combination of monitoring and price-changing rates in
$(0,1)^2$. The true drawback to the restriction $\tilde{\sigma}_i
\in PF(\alpha_{\tau},\beta_{\tau},\cut{k})$ lies in the possibility of
misspecifying equilibrium strategies, hence the importance of
establishing good fit of the policy function in
Section~\ref{ss:Policy_Fit}.

Additional technical details on the estimation of the structural
parameters have been relegated to the appendix, allowing us to turn
directly to the results.

%----------------------------------------------------------------------

\SubSection{Results}
\label{ss:Structural_Results}

Table~\ref{t:structural_estimates} presents the estimates of the
structural parameters from the profit equation \eqref{profit}. Recall
that we rely on estimates of $\Base_{it}$ and $\Upsell_{it}$ from
previous research, leaving managerial costs $\mu_\tau$ and $\chi_\tau$
as the only structural parameters to be estimated. We measure
$\mu_\tau$ and $\chi_\tau$ in dollars.

Type~1 firms are estimated to have a substantial monitoring cost of
around \$72, significantly different from 0 at better than the 5\%
level. While at first glance this estimate may seem high, further
consideration suggests it is plausible. The monitoring cost covers a
series of managerial activities including reviewing recent sales and
inputs, acquiring competition status, and integrating information from
several sources. If these activities occupy about an hour of the
manager's time, the monitoring cost should be comparable to the
manager's hourly pay. The estimated cost of changing price for a
type~1 firm is in fact negative, $-6.3$, although insignificantly
different from 0. Using the 95\% confidence interval, we can reject
that costs of changing price exceed \$4.1 at better than the 5\%
level. The estimates suggest that managers find it costly to
continually attend to the market but once they do attend, the menu
costs of electronically updating the price are fairly trivial. This
estimate confirms the notion that the advent of e-commerce can
practically eliminate physical menu costs. The stark contrast between
the two types of managerial costs are in align with a series of recent
macro papers. For example, Alvarez, Lippi and Paciello (2015) finds
the cost of reviewing information is three times as large as the cost
of changing price, and Zbaracki, Ritson, Levy, Dutta and Bergen (2004)
finds the managerial cost behind the decision of price change is about
six times as large as the physical cost of the actual price change.

Turning to type~2 firms, the estimate of the monitoring cost at \$48
is a bit lower though qualitatively similar to that for type~1. The
42.4 estimate for the cost of changing price is substantially higher
than that for type~1's. However, the remarkably wide confidence
interval $[-177.2, 145.8]$ around the estimate indicates that the cost
estimate is essentially uninformative. The small number (eight) of
type~2 firms and the infrequency of their price changes combine to
generate few price-change observations that go into identifying the
policy function. This shows up in standard errors for the
policy-function parameters in Table~\ref{t:policy} that are around
five times higher for type~2 than type~1 firms, leading to the wide
confidence interval seen here. The clustering procedure that divided
firms up into types allowed us to keep heterogeneous type~2 firms from
contaminating the estimates for other types but did not provide enough
data to allow credible estimation of that type's managerial costs.

The estimates for type~3 firms are almost identical to those for
type~1. They also have a substantial monitoring cost of around \$68
and negative but insignificant cost of changing price of $-7.4$. One
might have thought the differences between type~1 and type~3 firms
stem from differences in managerial capacity. Perhaps type~1 managers
occupy the low ranks because they can cheaply monitor and respond to
this active segment of the market. In fact, while type~1 and type~3
firms occupy different ranks, they do not differ much in their
price-changing behavior, as Figure~\ref{f:distributions} showed. This
similarity naturally translates into similar managerial costs. One
possible story is that similar managers are indifferent among various
rank positions because high margins and upselling profits (per
consumer) at high ranks balance low sales. The indifferent managers
are content then to spread and fill out the rank space.

Our preferred estimates given in the bottom of
Table~\ref{t:structural_estimates} re-estimate the structural
parameters after imposing a non-negativity constraint on costs. This
is a compelling theoretical restriction because it is hard to imagine
monitoring or inputting prices providing a utility boost. Perhaps more
importantly, the assumption plays an essential role in separately
identifying the components of managerial costs in certain cases, in
particular when the policy function happens to generate a price change
for almost all monitoring episodes. To understand this essential role,
consider the limiting case in which the manager changes price every
time he monitors. Then $\mu_\tau = 100$ and $\chi_\tau = 0$ would
produce the same net profit as $\mu_\tau = 1,100$ and $\chi_\tau =
-1,000$, as indeed would every linear combination of $\mu_\tau$ and
$\chi_\tau$ on the line including these points. The structural
parameters would be unstable and huge positive values for $\mu_\tau$
and huge negative values for $\chi_\tau$ could result. Imposing the
constraint $\mu_\tau, \chi_\tau \geq 0$ eliminates this instability
and selects plausible values of the structural
parameters.\footnote{Although our estimated policy functions happen to
  generate substantially more monitoring episodes than price changes,
  for some bootstrapped samples, the estimated policy functions
  generate similar numbers of monitors and price changes, leading to
  the problem of wildly positive values of $\mu_\tau$ and negative
  values of $\chi_\tau$. To address this problem in the unconstrained
  estimation, for a valid boostrap draw, we required the number of
  monitors to exceed the number of price changes by at least 5\%. The
  percentage of valid bootstraps is listed in
  Tabel~\ref{t:structural_estimates}. A side benefit of imposing the
  non-negativity constraint is that it eliminates the problem of
  invalid bootstraps, as indicated by 100\% of the bootstrapped sample
  being valid, as stated at the bottom of the table.}  The
non-negativity constraint binds for type~1 and~3's cost of changing
price. That estimate becomes precisely 0, while the cost of monitoring
falls slightly for them and its confidence interval tightens. The
estimates for type~2 remain unchanged although the 95\% confidence
interval around that type's monitoring cost now includes 0 and ranges
up to 210.9, reinforcing the impression that the structural estimates
for that type are simply uninformative.

%----------------------------------------------------------------------

\SubSection{Sensitivity}
\label{ss:Structural_Sensitivity}

This section gauges the sensitivity of the results to several
specification choices made in constructing the profit and value
functions. Regarding monetary profits, we adopted a particularly
simple specification in equation \eqref{profit}, the markup of price
over unit cost times quantity, measuring unit cost as the wholesale
acquisition cost of a memory module using our daily data. In practice,
firms may have had other unit costs. Credit-card purchases involve a
transaction fee, typically 2.25\%. There is also a small chance of
loss or breakage of the wholesale product. Offsetting these extra
costs, firms gained extra revenue through the approximately \$3
difference between actual shipping costs and the \$9.99 they
commonly charged the consumers. Our presumption is that these extra
costs and revenues largely cancel each other out, leaving equation
\eqref{profit} as an accurate representation of monetary
profits. Table~\ref{t:robustness} shows how the structural parameters
change with alternative specifications of monetary profits.

The first row for both the cost of monitoring and price change repeats
the baseline estimates from the previous table for comparison. The
structural costs presented have all been estimated imposing the
non-negativity constraint. Following the baseline estimates, the next
row recomputes the structural estimates after adding \$2 to unit
cost. In general, higher unit costs lead the estimates of managerial
costs to fall.  Intuitively, the higher are unit costs, the lower are
unit profits; lower managerial costs are required to justify the
observed frequency of managerial activity. Qualitatively, though, the
results do not change much, as the monitoring cost still dominates the
cost of price changing for type~1 and type~3 firms. Indeed, the cost
of price changing continues to hit the zero lower bound for type~1 and
type~3 firms but it remains positive for type~2 firms.

Our specification of monetary profits used the function in
Figure~\ref{f:rank}B for upselling profits. The next row for each
managerial cost re-estimates the structural parameter cutting the
assumed upselling profits in half for that type. Again, as expected,
this leads to lower estimates for managerial costs.  For example,
monitoring costs for type~1 firms falls from \$68.4 to \$26.0. While
this is a large quantitative change, still the qualitative picture
remains that for type~1 and type~3 firms, the monitoring cost is
substantial while the cost of price changing is negligible. Notably,
after this reduction in upselling profits, type~2 firms start to look
much like the others. The cost of price changing hits the zero lower
bound for them, too. Therefore, we cannot be sure whether the
inactivity of type~2 firms is explained by a high cost of changing
price or a lack of add-on profits. Some anecdotal evidence for the
latter interpretation comes from examining firms' product
webpages. While it is quite common for type~1 and type~3 firms to have
complex webpages emphasizing available upgrades, most type~2 firms
have very simple webpage designs offering little information on
premium options.

The last row re-estimates the structural parameters changing the
profit specified for firms that disappear from our sample of the first
two Pricewatch pages because they rise to rank 25 or higher. As
mentioned in footnote~\ref{fn:25}, this was set to \$4.70, but here we
cut it in half. This causes very little change in the estimated
managerial cost for type~1 firms because they spend very little time
at high ranks. For types-2 and 3, estimated monitoring costs rise to
offset the higher benefit from adjusting price to avoid falling off
the list. For all three types, the qualitative results are insensitive
to this change in specification.

%%%%%%%%%%%%%%%%%%%%%%%%%%%%%%%%%%%%%%%%%%%%%%%%%%%%%%%%%%%%%%%%%%%%%%%

\Section{Counterfactuals}
\label{s:Counterfactuals}

To assess the importance of managerial costs for the firm's pricing
behavior, in this section we present counterfactual exercises in which
we shock a single firm $i$'s managerial costs and see how its pricing
behavior and profit change, leaving other firms the same as before. In
effect, this section performs the inverse exercise from the structural
estimation in the previous section. In the structural estimation, we
estimated a policy function from actual pricing behavior and used this
to infer firms' managerial costs. Here, we posit a vector of
managerial costs for $i$ and search over policy functions for the one
maximizing its simulated profits. For each candidate policy function,
we want to simulate $i$'s new pricing behavior and profit assuming
that all other firms continue with their originally estimated
managerial costs, equivalent to assuming that their pricing behavior
is given by the originally estimated policy functions. This is exactly
the simulation we did for the 1,800 deviations during the structural
estimation, so we can treat these deviations as candidate policy
functions, and use the recorded simulations to recalculate simulated
profits with the new managerial costs for these deviations, and pick
the one with the highest profits as the firm's optimal policy at the
new managerial costs.  In fact, for this purpose, we did a much finer
search by simulating for additional several thousands candidate policy
functions that are drawn sequentially from the proximity of the
existing promising ones.  To save space we will focus on the case in
which $i$ is a type~1 firm. The results are shown in
Figure~\ref{f:counterfactuals}.

Panel~A shows the monthly number of prices changes for the firm as the
costs of monitoring and price change vary from 0 to 100 dollars. With
a large yet finite pool of high quality candidate policy functions, we
are able to identify 31 policy functions as optimal for some given
vectors of managerial costs. The surface is therefore marked by 31
plateaus, which indicate regions of costs for which we identified the
same strategy as optimal. Even as costs of monitoring and price
changing go to 0, the frequency of price change does not grow without
bound. In other words, in the complete absence of frictions associated
with price change, it is not optimal for a firm to continuously tweak
its price.  There are three reasons for this.  First, other firms have
retained their positive costs, so they do not respond
continuously. This results in stretches when state variables do not
change during which $i$'s optimal strategy is to keep price
constant. Second, changing price, especially downwards, may trigger
other firms' reaction and intensity future competition, so there is a
dynamic incentive to wait for a while between price changes. Third,
prices in this market are posted in whole dollar amounts, so even if a
firm continuously monitored its optimal continuous price, it still
would not want to change the price until the optimal price exceeded
the threshold necessary to move the price a whole dollar up or down.
When $i$ has no costs of monitoring or price changing, it ends up
changing price around once a day.

Panel~B explores the same counterfactual exercise but now focuses on a
different outcome variable: $i$'s monthly profits. These are the net
profits from equation \eqref{profit}, which subtract off the new
managerial costs with which we are shocking firm $i$, accumulated over
the month. While the surface in the previous panel had discrete jumps
reflecting the discrete changes to firm $i$'s pricing policy, the
continuous changes in $i$'s costs smooth out its profit function in
this panel. We will highlight several key insights that can be drawn
from the graph. The graph shows that the division between the two
types of managerial costs matters for profit. At the estimated costs
of \$68.4 for monitoring and \$0 for price changing, the firm's net
profit is \$6,070.  Holding the total managerial cost of a joint
episode of monitoring and price changing constant at \$68.4 but
shifting more of the joint cost from monitoring to price changing, the
firm's monthly profits montonically increase from \$6,070 to
\$6,496. In principle, simple arithmetic could explain this gain:
holding constant the firm's monitoring and price-changing episodes, it
gains from shifting more of the joint cost onto price changing because
it monitors multiple times for each price change, so managerial costs
would be lower with free monitoring and \$68.4 price changing rather
than vice versa. In fact the gain here is due to a more subtle change
in the firm's strategy. When monitoring is expensive, the firm ends up
forgoing some prime opportunities to change price and changing price
is some other less than prime circumstances knowing that it would be
too expensive to delay and keep tabs on the market. As monitoring
becomes cheaper, the firm can keep almost continual tabs on the market
and change price in exactly the right states. At the estimated costs
of \$68.4 for monitoring and \$0 for price changing, the firm's
optimal strategy leads it to monitor an average of 10.7 times per
month, generating managerial costs of $68.4 \cdot 10.7 = 731.9$. If
the costs are shifted to \$0 for monitoring and \$68.4 for price
changing, the firm 's optimal policy now leads it to continually
monitor and change price 12.1 times a month, generating managerial
costs of $68.4 \cdot 12.1 = 827.6$. Aggregate managerial costs are
thus higher with more weight shifted from monitoring to price
changing. So it is not an arithmetic reduction in cost that leads to
the counterfactual profit increase. Rather this increase in profit
comes from improvements in the firm's pricing policy, possible when
the managerial activity with additional option
value---monitoring---becomes cheaper even as the managerial activity
without option value---price changing---becomes more expensive.

Another key insight from Panel~B, which can be drawn from considering
the height of the surface, is the potentially large gain to adopting
technologies that decrease managerial costs, potentially thousands of
dollars a month just for this one product. In particular, based on
numbers in this graph, the monthly net profits would increase by over
\$1,500 if the managerial costs decreased from the estimated levels to
zero. One would underestimate the profits improvement assuming the
firm simply saves the managerial cost, which is about a half of
\$1,500, without showing a policy response.  Re-optimizing the policy
function to a suitably active one contributes the other half of the
profit improvement.  The gain would presumably be multiplied if the
technology could be used to reduce managerial costs for the scores of
other products the retailers marketed on Pricewatch. In fact the
retailers did move to automated pricing soon after the time period of
our data, consistent with our estimates of potentially large gains
from doing so.

%%%%%%%%%%%%%%%%%%%%%%%%%%%%%%%%%%%%%%%%%%%%%%%%%%%%%%%%%%%%%%%%%%%%%%%

\Section{Conclusion}
\label{s:Conclusion}

In this paper, we studied firms' price-changing behavior in an online
market for computer components. Special features of this market made
it particularly suitable for study: firms were ranked according to
price with lower-price firms receiving more prominent listings and the
bulk of the sales on the market; this ranking system, coupled with
rapidly changing market conditions, gave firms an incentive to change
price frequently as they jockeyed for position. The abundance of
price-changing episodes over the year of high-frequency (hourly)
observations offers an opportunity to precisely estimate a structural
model of price-changing behavior.

We used some initial reduced-form evidence to direct the structural
modeling. While firms were free to change price continuously, as
frequent as the price changes were, they were far from
continuous. Despite competing in a nationally integrated market (at
least to some extent), managers on the East and West Coast changed
prices at times during the day suiting their shifted schedules, and
managers in all locations rarely changed prices on weekends. We took
this as reduced-form evidence that the costs of managerial activity
provided a source of pricing inertia, motivating a structural model
allowing us to separately estimate both the managerial cost of
monitoring the market and the managerial cost of entering the new
price (a pure menu cost).  We were also careful to incorporate firm
heterogeneity in the structural model, based on visual examination of
price and rank paths for some representative firms showing systematic
differences in their strategies, with some firms changing price at
least weekly while others only once or twice a month and with some
firms aggressively targeting prominent ranks while others higher ranks
with lower sales. We incorporated firm heterogeneity in the structural
estimates by using a machine-learning method to cluster sample firms
method into three types and then estimating policy functions and
structural parameters allowed to freely vary across types.

Given the emphasis on frictions in managerial behavior, and the
prohibitive complexity of the state space in our setting, we built a
dynamic model of a boundedly rational manager. The manager changes
price according to a rule of thumb based on a subset of state
variables attended to rather than making the price change that would
be optimal given the full state space each instant. The key step for
our approach to work is to estimate a rule of thumb---in the language
of dynamic structural estimation, a policy function for price
changes---that accurately captures managerial behavior. We do this by
allowing for a sufficiently rich set of carefully selected state
variables, further enriched by allowing for different policy functions
across heterogeneous firm types, and then determining that the policy
function performs well across a suite of goodness-of-fit
exercises. Following BBL, we estimate the structural
parameters---here, the managerial costs of monitoring and changing
price---as those rationalizing the estimated policy function as being
more profitable than deviations. The new feature added to accommodate
bounded rationality is that deviations are restricted to the class of
admissible rules of thumb rather than any arbitrary price change.

For the types of firms for which we have reliable estimates (types 1
and 3), we estimate a cost of monitoring of roughly \$60 and
essentially no cost of price changing. (The estimates for type~2 firms
are considerably noisy because there are relatively few of these firms
and they changed price infrequently, leaving few price-changing
episodes to use to estimate the model.) While these managerial-cost
estimates can be moved around by introducing or removing factors in
the profit function, the qualitative results remain. Managerial costs
primarily arise in the monitoring stage; the further cost of inputting
price changes is fairly trivial, suggestive of technological features
of this e-commerce market.

This paper fills several gaps in the economics literature.  First, we
extend methods in the literature on dynamic structural estimation to
accommodate behavioral agents who may behave according to rules of
thumb based on a subset of state variables. This is an important
extension in our setting, where our central focus is on limits to
managerial capacity, but may be realistic in other settings as
well. Second, we contribute to the empirical macroeconomics literature
on retail price stickiness. Like that literature, we suggest that the
costs of managerial attention and activity may be important.  Our
novel contribution is a carefully specified dynamic model of pricing
behavior that can generate actual structural estimates of these costs.
Finally, we contribute to the behavioral-economics literature by,
first, providing a framework for structural estimation in the presence
of behavioral agents and, second, by providing numerical estimates of
the cost of managerial activity, providing an insight into the
psychological barriers of actions such as changing prices that are
typically regarded as automatic in neoclassical models.

%%%%%%%%%%%%%%%%%%%%%%%%%%%%%%%%%%%%%%%%%%%%%%%%%%%%%%%%%%%%%%%%%%%%%%%%%%
\clearpage

\end{spacing}

\setcounter{equation}{0}
\renewcommand{\theequation}{A\arabic{equation}}

\section*{Appendix A: Details on Structural Estimation}

This appendix provides additional technical details on the estimation
of the structural parameters omitted from the text.

To transform our identification condition \eqref{identification_ours}
into an estimator of $\theta$ requires empirical analogs to the
expectations over value functions appearing there, which we will
compute via simulation. Our empirical analogue to the first
expectation, $E_{s\in\hat{S}} [ V_{i}(s;\hat{\sigma},\theta )]$, is
\begin{equation}
\label{sim_eq}
\widehat{EV}_i (\hat{\sigma},\theta ) = \frac{1}{MN} \sum_{m = 1}^M
\sum_{n=1}^N \sum_{t_{mn}=0}^{\min\{T,720\}} \delta^{t_{mn}}
\pi(s_{t_{mn}}, \hat{\sigma}(s_{t_{mn}}),\theta).
\end{equation}
There are three summations in \eqref{sim_eq}. The first sum simulates
the expectation over initial states represented by the
$E_{s\in\hat{S}}$ operator. We do this by taking $M$ draws from the
set of state vectors observed in the data, $\hat{S}$, and averaging
the result (hence the division by $M$). The second sum simulates the
expectation implicit in the value function $V_i$; this expectation is
over the distribution of all possible histories of game play and
private shocks starting from the given initial state. We compute this
expectation by simulating $N$ histories for each of the $M$ initial
states and averaging the resulting value functions (hence the division
by $N$). The third sum adds up the profit stream implicit in the value
function.

The other expectation, $E_{s\in\hat{S}} [V_i(s;\tilde{\sigma}_i,
  \hat{\sigma}_{-i}, \theta )]$ from the identification condition
\eqref{identification_ours} is similarly transformed into its
empirical analogue $\widehat{EV}_i (s;\tilde{\sigma}_i,
\hat{\sigma}_{-i}, \theta )$. Instead of all firms behaving according
to the estimated policy function $\hat{\sigma}$, firm $i$ deviates to
the policy function $\tilde{\sigma}_i \in
PF(\alpha_{\tau},\beta_{\tau},C_k)$ in the simulation, resulting in
profit $\pi(s_{t_{mn}}, \tilde{\sigma}_i(s_{t_{mn}}),
\hat{\sigma}_{-i}(s_{t_{mn}}),\theta)$ as the new summand in
\eqref{sim_eq}.

The upper limit in the third sum is modified from the value function
as originally appears in \eqref{disprofit}. Instead of calculating the
value function over an indeterminate number of periods ending with
firm $i$'s exit at time $T_i$, we just add up the profit during the
first month---720 hours to be precise. We do this to reduce the
accumulation of simulation errors as the period becomes longer. Given
the nature of deviations we are considering, with the firm deviating
to a whole new policy function for the entire game, profits are
stationary in all simulations. Hence the average per-period profit
over 720 hours is an unbiased estimate of the average over the whole
game. This is not true of some excluded deviations, for example, a
one-time increase in price of \$1; such a deviation could generate a
complicated impulse response, which would lead the average profit over
a truncated period to diverge from that over the full game. We set the
discount factor to an annual value of 0.95, though discounting turns
out to be fairly inconsequential over the month horizon we are
considering.

Calculating $\widehat{EV}_i$ is further expedited following BBL's
insight that when the profit function is linear in the structural
parameters, this linearity is inherited by $\widehat{EV}_i$ because it
is essentially an average over these linear profits. Thus, for
example, we can write $\widehat{EV}_i (s;\hat{\sigma},\theta)$ as the
dot product of two vectors
\begin{equation}
\label{linearity}
\widehat{EV}_i (s;\hat{\sigma},\theta) = \vv{EV}_i(s;\hat{\sigma})
\cdot \vv{\theta}_{\tau}.
\end{equation}
Here $\vv{EV}_i(s;\hat{\sigma})$ is a vector with four components,
corresponding to the four terms in the profit function in
\eqref{profit}. The first component is the sum of the stream of firm
$i$'s base profits in a simulation, $\sum_{t_{mn}=0}^{\min\{T,720\}}
\Base_{it_{mn}}$, averaged over the $MN$ simulations, where firms
behave according to their estimated policy functions in each
simulation. Similarly, the remaining three components are the averages
over the $MN$ simulations of, respectively, the total upselling profit
over a simulation, $\sum_{t_{mn}=0}^{\min\{T,720\}}
\Upsell_{it_{mn}}$; the number of times firm $i$ monitored market
conditions during a simulation, $\sum_{t_{mn}=0}^{\min\{T,720\}}
\Monitor_{it_{mn}}$; and the number of times firm $i$ changed price
during a simulation, $\sum_{t_{mn}=0}^{\min\{T,720\}}
1\{\Delta_{it_{mn}} \neq 0\}$. The second vector in \eqref{linearity}
includes the structural parameters for a firm of type $\tau$: i.e.,
$\vv{\theta}_{\tau} = (1, \upsilon_{\tau}, \mu_{\tau}, \chi_{\tau})$.
The linearity in \eqref{linearity} means that the summary statistics
$\vv{EV}_i(s;\hat{\sigma})$ are all that need to be saved from the
$MN$ simulations to later compute $\widehat{EV}_i
(s;\hat{\sigma},\theta)$ for any given $\theta$ because one just needs
to take the dot product of the summary statistics and subvectors of
the given $\theta$. Hence the simulation and estimation steps can
essentially be conducted independently.

The other expectation estimate can be expressed similarly as
\begin{equation}
\label{linearity2}
\widehat{EV}_i (s;\tilde{\sigma}_i, \hat{\sigma}_{-i}, \theta ) =
\vv{EV}_i(s; \tilde{\sigma}_i, \hat{\sigma}_{-i}) \cdot
\vv{\theta}_{\tau},
\end{equation}
in this case conducting the simulations with firm $i$ using its
deviation strategy $\tilde{\sigma}_i$ in the simulation while other
firms use their estimated policy functions.

With these estimates of the expectations in identification condition
\eqref{identification_ours} in hand, we can proceed to estimation of
the structural parameters $\theta$.  Following BBL, define $Q$ to be
the change in the value function caused by a deviation in strategy:
\begin{eqnarray}
\label{Q}
Q(\hat{\sigma}, \tilde{\sigma}_i, \theta_{\tau}) & = & \left[ \widehat{EV}_i (s;\hat{\sigma},\theta) -  \widehat{EV}_i (s;\tilde{\sigma}_i, \hat{\sigma}_{-i}, \theta ) \right] \label{Q1} \\
& = & \left[ \vv{EV}_i(s;\hat{\sigma}) - \vv{EV}_i(s; \tilde{\sigma}_i, \hat{\sigma}_{-i}) \right] \cdot \vv{\theta}_{\tau}, \label{Q2}
\end{eqnarray}
where \eqref{Q2} follows from \eqref{linearity} and
\eqref{linearity2}. The force of identification assumption
\eqref{identification_ours} here is that $Q$ will be non-negative for
sufficiently accurate estimates of the expectations and for $\theta$
sufficiently close to the true structural parameters.  We will
estimate $\theta$ by assessing a penalty for violations of
non-negativity and choosing the value that minimizes the sum of
squared penalties over the deviations considered:
\begin{equation}
\label{penalty}
\hat{\theta} = \argmin_{\{\vv{\theta}_{\tau} | \tau = 1,2,3\}} \left\{
\sum_{\{ \tilde{\sigma}_i | \tau = 1,2,3 \}} \left[ \min \{
  Q(\hat{\sigma}, \tilde{\sigma}_i, \theta_{\tau}) , 0\} \right]^2
\right\}.
\end{equation}

For each firm type, we considered a set of 2,000 deviations $\sigma_i$
from the set of policy functions
$PF(\alpha_{\tau},\beta_{\tau},\cut{k})$. Each deviation was generated
by adding three perturbations to the type's estimated policy
function. The first perturbation involves pure random noise: we
multiplied the coefficients $\hat{\alpha}_{\tau}$,
$\hat{\beta}_{\tau}$ by log-uniform noise terms and shifted the cut
points $\hat{C}^k_\tau$ by uniform noise terms. The second
perturbation involves adding correlated noise terms to the constant
term in $\hat{\alpha}_{\tau}$ and the cut points $\hat{C}^k_\tau$,
such that the deviations amount to an experiment with changing the
frequency of monitoring and frequency of price change conditional on
monitoring in opposite directions. The third perturbation is to add
correlated noise terms to the cut points $\hat{C}_k$, such that the
deviations amount to experiments with trade off between small and
large price changes.  We chose these perturbations to reflect actual
tradeoffs that firms might face in their decision-making.  For each
deviation $\tilde{\sigma}_i$, we calculate the value function
statistics $\vv{EV}_i(s; \tilde{\sigma}_i, \hat{\sigma}_{-i})$ with $M
= 10,000$ random draws of initial states with replacement and $N=1$
simulation for each initial state.  For the estimated policy, we
calculate the value function statistics $\vv{EV}_i(s;\hat{\sigma})$
with $M = 1,000,000$ random draws of initial states with replacement
and $N=1$ simulation for each initial state to attain greater
accuracy.

Because we have chosen a large number of deviations and a large number
$MN$ of simulations for each deviation, the second-stage estimate,
$\hat{\theta}$, is quite accurate conditional on the first-stage
estimate $\hat{\sigma}$. The main source of randomness in
$\hat{\theta}$ therefore is the potential error in the first stage. To
account for this main source of error, we create 200 bootstrapped
samples and repeat the whole estimation procedure.
For each
  bootstrapped sample $n = 1, \ldots, 200$, we derive a new estimate,
  $\hat{\sigma}^{(n)}$, of the policy function. For each $n$, we carry
  out the second-stage using $\hat{\sigma}^{(n)}$ in the place of
  $\hat{\sigma}$ in \eqref{Q}, \eqref{Q2} and \eqref{penalty} and
  generate new sets of deviations by perturbing $\hat{\sigma}^{(n)}$.
  The resulting structural estimates $\hat{\theta}^{(n)}$, to minimize
  the sum of squared penalties represented by the new $Q$
  function. Confidence intervals can be constructed from quantiles of
  the set $\{\hat{\theta}^{(n)} | n = 1,\dots, 200 \}$. We could have
  used bootstrapping to compute standard errors for the
  policy-function parameters $\alpha_\tau$, $\beta_\tau$, and
  $\cut{k}$ reported in Table~\ref{t:policy}. It turns out that these
  bootstrapped standard errors are quite close to the standard errors
  from the maximum-likelihood procedure we chose to report. To
simplify the estimation, we did not incorporate uncertainty in
$Q(\Rank_{it})$ and $U(\Rank_{it})$ in our standard errors, but we
note that those expressions were fairly precisely estimated in Ellison
and Ellison (2009a).

%%%%%%%%%%%%%%%%%%%%%%%%%%%%%%%%%%%%%%%%%%%%%%%%%%%%%%%%%%%%%%%%%%%%%%%

\newpage

\section*{References}

\begin{description}

\frenchspacing

\item Aguirregabiria, Victor and Pedro Mira. (2007) ``Sequential
  Estimation of Dynamic Discrete Games,'' {\it Econometrica} 75:
  1--53.

\item Alvarez, Fernando E., Francesco Lippi, and Luigi
  Paciello. (2011) ``Optimal Price Setting with Observation Menu
  Costs,'' {\it Quarterly Journal of Economics} 126: 1909--1960.

\item Arbatskaya, Maria and Michael Baye. (2004) ``Are Prices `Sticky'
  Online?  Market Structure Effects and Asymmetric Responses to Cost
  Shocks in Online Mortgage Markets,'' {\it International Journal of
    Industrial Organization} 22: 1443--1462.

\item Artinger, Florian and Gerd Gigerenzer. (2012)
  ``Aspiration-adaption, Price Setting, and the Used Car Market,''
  mimeo.

\item Atkinson, Benjamin. (2009) ``Retail Gasoline Price Cycles:
  Evidence from Guelph, Ontario Using Bi-Hourly, Station-Specific
  Retail Price Data,'' {\em Energy Journal} 30: 85--110.

\item Bajari, Patrick C., Lanier Benkard, and Jonathan Levin. (2007)
  ``Estimating Dynamic Models of Imperfect Competition,'' {\it
  Econometrica} 75: 1331--1370.

\item Barron, John M., Beck A. Taylor, and John R. Umbeck. (2004)
  ``Number of Sellers, Average Prices, and Price Dispersion,'' {\it
  International Journal of Industrial Organization} 22: 1041--1066.

\item Baye, Michael R., John Morgan, and Patrick Scholten. (2004)
  ``Price Dispersion in the Small and in the Large: Evidence from an
  Internet Price Comparison Site,'' {\it Journal of Industrial
    Economics} 52: 463--496.

\item Bils, Mark and Peter Klenow. (2004) ``Some Evidence on the
  Importance of Sticky Prices,'' {\it Journal of Political Economy}
  112: 947--985.

\item Blinder, Alan, Eile Canetti, David Lebow, and Jeremy
  Rudd. (1998) {\it Asking About Prices---A New Approach to
    Understanding Price Stickiness}.  New York: Russell Sage
  Foundation.

\item Bonomo, Marco, Carlos Carvalho, Ren� Garcia, and Vivian
  Malta. (2015) ``Persistent Monetary Non-neutrality in an Estimated
  Model with Menu Costs and Partially Costly Information,'' Society
  for Economic Dynamics 2015 meetings paper no.~1339.

\item Borenstein, Severin, A. Colin Cameron, and Richard
  Gilbert. (1997) ``Do Gasoline Prices Respond Asymmetrically to Crude
  Oil Price Changes?'' {\it Quarterly Journal of Economics} 112:
  305--339.

\item Carlton, Dennis. (1986) ``The Rigidity of Prices,'' {\it
  American Economic Review} 76: 637--658.

\item Castanias, Rick and Herb Johnson. (1993) ``Gas Wars: Retail
  Gasoline Price Fluctuations,'' {\it Review of Economics and
    Statistics} 75: 171--174.

\item Cavallo, Alberto and Roberto Rigobon. (2012) ``The Distribution
  of the Size of Price Changes,'' National Bureau of Economic Research
  Working Paper no.~w16760.

\item Chakrabarti, Rajesh and Barry Scholnick. (2005) ``Nominal
  Rigidities without Literal Menu Costs: Evidence from E-Commerce,''
  {\it Economics Letters} 86: 187--191.

\item Clay, Karen, Ramayya Krishnan, Eric Wolff, and Danny
  Fernandes. (2002) ``Retail Strategies on the Web: Price and
  Non-price Competition in the Online Book Industry,'' {\it Journal of
    Industrial Economics} 50: 351--367.

\item Cyert, Richard and James March. (1963) {\it Behavioral Theory of
  the Firm}.  Oxford: Blackwell Publishers.

\item Davis, Michael C. and James D. Hamilton. (2004) ``Why Are Prices
  Sticky? The Dynamics of Wholesale Gasoline Prices,'' {\it Journal of
    Money, Credit and Banking} 36: 17--37. 

\item Doyle, Joseph, Erich Muehlegger, and Krislert
  Samphantharak. (2010) ``Edgeworth Cycles Revisited,'' {\em Energy
    Economics} 32: 651--660.

\item Eckert, Andrew and Douglass West. (2004) ``Retail Gasoline Price
  Cycles across Spatially Dispersed Gasoline Stations,'' {\it Journal
    of Law and Economics} 47: 245--271.

\item Edgeworth, Francis. (1925) ``The Pure Theory of Monopoly,'' in
  {\it Papers Relating to Political Economy}, vol.~1. London:
  MacMillan, 111--142.

\item Eichenbaum, Martin, Nir Jaimovich, and Sergio Rebelo. (2011)
  ``Reference Prices, Costs, and Nominal Rigidities,'' {\it American
  Economic Review} 101: 234--262

\item Ellison, Glenn and Sara Fisher Ellison. (2009a) ``Search,
  Obfuscation, and Price Elasticities on the Internet,'' {\it
    Econometrica} 77: 427--452.

\item Ellison, Glenn and Sara Fisher Ellison. (2009b) ``Tax
  Sensitivity and Home State Preferences in Internet Purchasing,''
  {\it American Economic Journal: Economic Policy} 1: 53--71.

\item Goldberg, Pinelopi K. and Rebecca Hellerstein. (2013) ``A
  Structural Approach to Identifying the Sources of Local Currency
  Price Stability,'' {\it Review of Economic Studies} 80: 185--210.

\item Golosov, Mikhail and Robert E. Lucas Jr. (2007) ``Menu Costs and
  Phillips Curves,'' {\it Journal of Political Economy} 115: 171--199.

\item Gorodnichenko, Yuriy and Michael Weber. (2015) ``Are Sticky
  Prices Costly? Evidence from the Stock Market,'' {\it American
    Economic Review} 106: 165--199.

\item Harris, Mark N. and Xueyan Zhao. (2007) ``A Zero-Inflated
  Ordered Probit Model, with an Application to Modelling Tobacco
  Consumption,'' {\it Journal of Econometrics} 141: 1073--1099.

\item Heckman, James. (1979) ``Sample Selection Bias as a
  Specification Error,'' {\em Econometrica} 47: 153--161.

\item Hosken, Daniel S., Robert S. McMillan, and Christopher
  T. Taylor. (2008) ``Retail Gasoline Pricing: What Do We Know?'' {\it
    International Journal of Industrial Organization} 26: 1425--1436.

\item Klenow, Peter J. and Olesksiy Kryvtsov. (2008) ``State-Dependent
  or Time-Dependent Pricing: Does it Matter for Recent
  U.S. Inflation?'' {\it Quarterly Journal of Economics} 123:
  863--904.

\item Klenow, Peter J. and Benjamin A. Malin. (2011) ``Microeconomic
  Evidence on Price-Setting,'' in Benjamin Friedman and Michael
  Woodford, eds., {\it Handbook of Monetary Economics},
  vol.~3. Amsterdam: North-Holland, 231--284.

\item Lach, Saul. (2002) ``Existence and Persistence of Price
  Dispersion: An Empirical Analysis,'' {\it Review of Economics and
    Statistics} 84: 433--444.

\item Lewis, Matthew. (2008) ``Price Dispersion and Competition with
  Differentiated Sellers,'' {\it Journal of Industrial Economics} 56:
  654--678.

\item Lewis, Matthew. (2009) ``Temporary Wholesale Gasoline Price
  Spikes Have Long-Lasting Retail Effects: The Aftermath of Hurricane
  Rita,'' {\it Journal of Law and Economics} 52: 581--605.

\item L\"{u}nneman, Patrick and Ladislav Wintr. (2006) ``Are Internet
  Prices Sticky?,'' ECB Working Paper no.~645.

\item Maskin, Eric and Jean Tirole. (1988) ``A Theory of Dynamic
  Oligopoly, II: Price Competition, Kinked Demand Curves, and
  Edgeworth Cycles,'' {\it Econometrica} 56: 571--599.

\item Means, Gardiner. (1935) ``Industrial Prices and Their Relative
  Inflexibility,'' U.S. Senate Document 13, 74th Congress, 1st
  Session.

\item Midrigan, Virgiliu. (2011) ``Menu Costs, Multiproduct Firms, and
  Aggregate Fluctuations,'' {\it Econometrica} 79: 1139--1180.

\item Nakamura, Emi and J\'on Steinsson.  (2008) ``Five Facts About
  Prices: A Reevaluation of Menu Cost Models,'' {\it Quarterly Journal
    of Economics} 123: 1415--1464.

\item Nakamura, Emi and Dawit Zerom. (2010) ``Accounting for
  Incomplete Pass-Through,'' {\it Review of Economic Studies} 77:
  1192--1230.

\item Noel, Michael. (2007a) ``Edgeworth Price Cycles, Cost-Based
  Pricing, and Sticky Pricing in Retail Gasoline Markets,'' {\it
    Review of Economics and Statistics} 89: 324--334.

\item Noel, Michael. (2007b) ``Edgeworth Price Cycles: Evidence from
  the Toronto Retail Gasoline Market,'' {\it Journal of Industrial
    Economics} 55: 69--92.

\item Noel, Michael. (2008) ``Edgeworth Price Cycles and Focal Prices:
  Computational Dynamic Markov Equilibria,'' {\em Journal of Economics
    and Management Strategy} 17: 345--377.

\item Pakes, Ariel, Ostrovsky, Michael, and Steven Berry. (2007)
  ``Simple Estimators for the Parameters of Discrete Dynamic Games
  (with Entry/Exit Examples),'' {\em Rand Journal of Economics} 38:
  373--399.

\item Pakes, Ariel, Jack Porter, Katherine Ho, and Joy Ishii. (2015)
  ``Moment Inequalities and Their Application,'' {\it Econometrica}
  83: 315--334.

\item Romesburg, H. Charles. (2004) {\it Cluster Analysis for
  Researchers}. North Carolina: Lulu Press.

\item Simon, Herbert. (1955) ``A Behavioral Model of Rational
  Choice,'' {\it Quarterly Journal of Economics} 69: 99--118.

\item Simon, Herbert. (1962) ``New Developments in the Theory of the
  Firm,'' {\it American Economic Review} 52: 1--15.

\item Slade, Margaret E. (1998) ``Optimal Pricing with Costly
  Adjustment: Evidence from Retail-Grocery Prices,'' {\it Review of
    Economic Studies} 65: 87--107.

\item Stahl, Dale O. (1989) ``Oligopolistic Pricing with Sequential
  Consumer Search,'' {\it American Economic Review} 79: 700--712.

\item Stigler, George J. and James K. Kindahl. (1970) {\it The
  Behavior of Administered Prices}. New York: Columbia University
  Press.

\item Terkel, Studs. (1970) {\it Hard Times: An Oral History of the
  Great Depression}. New York: Pantheon Books.

\item Vuong, Quang H. (1989) ``Likelihood Ratio Tests for Model
  Selection and Non-Nested Hypotheses,'' {\em Econometrica} 57:
  307--333.

\item Wang, Zhongmin. (2009) ``(Mixed) Strategy in Oligopoly Pricing:
  Evidence from Gasoline Price Cycles Before and Under a Timing
  Regulation,'' {\it Journal of Political Economy} 117: 987--1030.

\item Zbaracki, Mark J., Mark Tison, Daniel Levy, Shantanu Dutta, and
  Mark Bergen. (2004) ``Managerial and Customer Costs of Price
  Adjustment: Direct Evidence from Industrial Markets,'' {\it Review
    of Economics and Statistics} 86: 514--533.

\end{description}

%%%%%%%%%%%%%%%%%%%%%%%%%%%%%%%%%%%%%%%%%%%%%%%%%%%%%%%%%%%%%%%%%%%%%%%
\clearpage

\begin{sidewaystable}
\begin{footnotesize}
\begin{center}
\caption{\label{t:dstat}Variable Definitions and Descriptive Statistics}
\begin{tabular*}{\textwidth}{l @{\extracolsep{\fill}} l r r r r r}
\mc{7}{c}{ }\\
\hline \hline
\mc{7}{c}{ }\\
Variable & Definition & \mc{1}{c}{Mean} & \mc{1}{c}{Std. dev.} & \mc{1}{c}{Min.} & \mc{1}{c}{Max.} & \mc{1}{c}{Obs.} \\
\mc{7}{c}{ }\\
\hline
\mc{7}{c}{ }\\
\mc{7}{l}{Firm-level variables} \\[1ex]
 \hspace{1em}\Density & Measure of density in price space of firms with nearby ranks & 0.60 & 0.40 & 0 & 3 & 111,276 \\
 \hspace{1em}\Margin & Percentage markup over wholesale cost, $100(\Price-\Cost)/\Cost$ & 1.01 & 5.65 & -20.50 & 20.38 & 111,276 \\
 \hspace{1em}\NumBump & Net number of ranks bumped since last price change & 1.30 & 3.40 & -22 & 21 & 111,276 \\
 \hspace{1em}\Placement & Placement between adjacent firms in price space & 0.58 & 0.42 & 0 & 1 & 111,276 \\
 \hspace{1em}\Price & Current listed price in dollars & 70.1 & 34.6 & 25.0 & 131.0 & 111,276 \\
 \hspace{1em}\QuantityBump & Relative change in hourly sales resulting from rank bump & --0.16 & 0.36 & --2.08 & 2.48 & 111,276 \\
 \hspace{1em}\Rank & Rank of listing in price-sorted order & 10.75 & 6.77 & 1 & 24 & 111,276 \\
 \hspace{1em}\RankOne & Indicates whether firm is at rank 1 & 0.06 & 0.24 & 0 & 1 & 111,276 \\
 \hspace{1em}\SinceChange & Hours since firm last changed price & 117.55 & 146.45 & 1 & 1,113 & 111,276 \\
\mc{2}{c}{ } \\
\mc{2}{l}{Market-level variables} \\[1ex]
 \hspace{1em}\Cost & Wholesale cost & 66.40 & 36.85 & 23 & 129 & 7,740 \\
 \hspace{1em}\CostTrend & Trend in \Cost\ over previous two weeks & --0.19 & 0.71 & --2.06 & 1.53 & 7,740 \\
 \hspace{1em}\CostVol & Volatility of \Cost\ over previous two weeks & 1.64 & 1.08 & 0.00 & 4.36 & 7,740 \\
 \hspace{1em}\Night & Indicates hour from midnight to 8 a.m. EST & 0.33 & 0.47 & 0 & 1 & 7,740 \\
 \hspace{1em}\Weekend & Indicates Saturday or Sunday & 0.29 & 0.45 & 0 & 1 & 7,740 \\
\mc{7}{c}{ }\\
\hline
\end{tabular*}
\end{center}
\vspace{-2ex}
\begin{spacing}{1}
\noindent Notes: Firm-level variables vary over indexes $i$ and
$t$. Market-level variables vary over $t$. For these variables,
descriptive statistics are for the time series of one observation per
period.
\end{spacing}
\end{footnotesize}
\end{sidewaystable}

%%%%%%%%%%%%%%%%%%%%%%%%%%%%%%%%%%%%%%%%%%%%%%%%%%%%%%%%%%%%%%%%%%%%%%%
\clearpage

\begin{table}
\begin{footnotesize}
\begin{center}
\caption{\label{t:hetero}Variable Means by Firm Type}
\begin{tabular*}{\textwidth}{l @{\extracolsep{\fill}} s s s s}
\mc{5}{c}{ }\\
\hline \hline
\mc{5}{c}{ }\\
 & \mc{1}{c}{Combined types} & \mc{1}{c}{Type 1} & \mc{1}{c}{Type 2} & \mc{1}{c}{Type 3} \\
Variable & \mc{1}{c}{(43 firms)} & \mc{1}{c}{(22 firms)} & \mc{1}{c}{(8 firms)} & \mc{1}{c}{(13 firms)} \\
\mc{5}{c}{ }\\
\hline
\mc{5}{c}{ }\\
\Density & 0.60 & 0.63 & 0.52 & 0.56 \\
\Margin & 1.01 & -0.72 & 1.24 & 6.20 \\
\NumBump & 1.30 & 1.00 & 2.12 & 1.64 \\
\Placement & 0.58 & 0.58 & 0.58 & 0.58 \\
\Price & 70.06 & 68.09 & 85.42 & 64.90 \\
\QuantityBump & -0.16 & -0.17 & -0.18 & -0.11 \\
\Rank & 10.75 & 7.78 & 13.37 & 18.06 \\
\RankOne & 0.06 & 0.09 & 0.02 & 0.00 \\
\SinceChange & 117.55 & 99.88 & 255.08 & 93.33 \\[2ex]
Observations & \mc{1}{c}{111,276} & \mc{1}{c}{71,460} & \mc{1}{c}{16,904} & \mc{1}{c}{22,912} \\
\mc{5}{c}{ }\\
\hline
\end{tabular*}
\end{center}
\vspace{-2ex}
\begin{spacing}{1}
\noindent Notes: Shown are means only of variables varying over
indexes $i$ and $t$. Column for combined firms repeats information
from Table~\ref{t:dstat} for comparison.
\end{spacing}
\end{footnotesize}
\end{table}

%%%%%%%%%%%%%%%%%%%%%%%%%%%%%%%%%%%%%%%%%%%%%%%%%%%%%%%%%%%%%%%%%%%%%%%
\clearpage

\begin{sidewaystable}
\begin{footnotesize}
\begin{center}
\caption{\label{t:policy} Maximum Likelihood Estimates of Policy Function}
\begin{tabular*}{\textwidth}{l @{\extracolsep{\fill}} c dd c dd c dd}
\mc{10}{c}{ }\\
\hline \hline
 \mc{10}{c}{ } \\
 & & \mc{2}{c}{Type 1 firms} & & \mc{2}{c}{Type 2 firms} & & \mc{2}{c}{Type 3 firms} \\ \cline{3-4} \cline{6-7} \cline{9-10} 
 \mc{10}{c}{ } \\
 Variable & & \mc{1}{c}{Coefficient} & \mc{1}{c}{Std.\ err.} & & \mc{1}{c}{Coefficient} & \mc{1}{c}{Std.\ err.} & & \mc{1}{c}{Coefficient} & \mc{1}{c}{Std.\ err.} \\
 \mc{10}{c}{ } \\
 \hline
 \mc{10}{c}{ } \\
 \mc{10}{l}{Monitoring estimates $\alpha_\tau$} \\[1ex]
 \hspace{1em} Constant & & -2.25\one & (0.10) & & -1.71\one & (0.54) & & -2.83\one & (0.25) \\
 \hspace{1em} \Night & & -0.68\one & (0.05) & & -0.84\one & (0.23) & & -1.30\one & (0.21) \\
 \hspace{1em} \Weekend & & -0.47\one & (0.04) & & -0.61\one & (0.20) & & -0.71\one & (0.11) \\
 \hspace{1em} \CostVol & & 0.04\one & (0.01) & & 0.04 & (0.06) & & 0.05\ten & (0.03) \\
 \hspace{1em} $\CostTrend^{\,+}$ & & 0.13\one & (0.04) & & 0.33\five & (0.15) & & 0.18 & (0.12) \\
 \hspace{1em} $\vert \CostTrend^{\,-} \vert$ & & 0.09\one & (0.03) & & 0.02 & (0.21) & & 0.24\one & (0.07) \\
 \hspace{1em} $\QuantityBump\ ^+$ & & 0.39\one & (0.08) & & 0.69\one & (0.26) & & 0.13 & (0.35) \\
 \hspace{1em} $\vert \QuantityBump\ ^- \vert$ & & 0.36\one & (0.05) & & 0.15 & (0.26) & & 0.20 & (0.19) \\
 \hspace{1em} $\ln \SinceChange$ & & 0.25\one & (0.05) & & -0.23 & (0.17) & & 0.47\one & (0.12) \\
 \hspace{1em} $(\ln \SinceChange)^2$ & & -0.05\one & (0.01) & & 0.02 & (0.02) & & -0.07\one & (0.02) \\
 \mc{10}{c}{ } \\
 \hline
 \mc{10}{c}{ } \\
 \mc{10}{l}{Price change estimates $\beta_\tau$} \\[1ex]
 \hspace{1em}  \CostTrend & & 0.11\five & (0.05) & & 0.25 & (0.22) & & -0.31\five & (0.13) \\
 \hspace{1em}  \CostChange & & 0.06\one & (0.01) & & 0.06 & (0.04) & & 0.09\one & (0.03) \\
 \hspace{1em}  \Margin & & -0.02\five & (0.01) & & 0.04 & (0.04) & & -0.11\one & (0.03) \\
 \hspace{1em}  \NumBump & & -0.05\one & (0.02) & & 0.07 & (0.05) & & -0.05 & (0.04) \\
 \hspace{1em}  $Density \times \NumBump$ & & -0.10\one & (0.04) & & -0.28\one & (0.08) & & -0.06 & (0.06) \\
 \hspace{1em}  \Placement & & 0.30\one & (0.08) & & 0.19 & (0.32) & & 0.07 & (0.17) \\
 \hspace{1em}  \Rank & & -0.04\one & (0.01) & & -0.09\five & (0.04) & & -0.05\five & (0.02) \\
 \hspace{1em}  \RankOne & & 0.62\one & (0.11) & & \mc{1}{c}{$a$} & \mc{1}{c}{$a$} & & \mc{1}{c}{$a$} & \mc{1}{c}{$a$} \\[2ex]
 Cutoff $C^{\,-1}_\tau$ & & -0.15 & (0.14) & & -1.80 & (0.48) & & -1.62 & (0.37) \\
 Cutoff $C^{\,1}_\tau$ & & 0.76\one & (0.14) & & 0.55 & (0.87) & & -0.44 & (0.43) \\
 \mc{10}{c}{ } \\
 \hline
 \mc{10}{c}{ } \\
 Log likelihood & & \mc{2}{c}{--6,174.2} & & \mc{2}{c}{--383.0} & & \mc{2}{c}{--1,337.9} \\
 Observations & & \mc{2}{c}{71,460} & & \mc{2}{c}{16,904} & & \mc{2}{c}{22,912} \\
 \mc{10}{c}{ }\\
\hline
\end{tabular*}
\end{center}
\vspace{-2ex}
\begin{spacing}{1}
\noindent Notes: Coefficients from maximum likelihood estimation of
equations \eqref{monitoring} and \eqref{pricechange} separately for
each firm type. Heteroskedasticity-robust standard errors clustered by
firm reported in parentheses.  $^a$Because most of the observations
with $\RankOne = 1$ are in group~1, equations estimated for groups~2
and~3 constrain \RankOne\ coefficient to be the same as estimated for
group~1, 0.62. Model includes cutoffs $C_{k}$ for $k \in \{ -5, -4,
-3, -2, -1, 1, 2, 3, 4, 5\}$; for space considerations we only report
$C_{-1}$ and $C_1$. Statistically significant in a two-tailed test at
the \ten 10\% level, \five 5\% level, \one 1\% level.
\end{spacing}
\end{footnotesize}
\end{sidewaystable}

%%%%%%%%%%%%%%%%%%%%%%%%%%%%%%%%%%%%%%%%%%%%%%%%%%%%%%%%%%%%%%%%%%%%%%%
\clearpage

\begin{table}
\begin{footnotesize}
\begin{center}
\caption{\label{t:vuong}Vuong Non-Nested Specification Tests}
\begin{tabular*}{\textwidth}{l @{\extracolsep{\fill}} d d d}
\mc{4}{c}{ }\\
\hline \hline
\mc{4}{c}{ }\\
& \mc{3}{c}{Versus alternative model} \\ \cline{2-4}
\mc{4}{c}{ }\\
Model & \mc{1}{c}{B. ZIOP shifting variables} & \mc{1}{c}{C. ZIOP omitting variables} & \mc{1}{c}{D. OP omitting monitoring} \\
\mc{4}{c}{ }\\
\hline
\mc{4}{c}{ }\\
A. ZIOP preferred specification & 3.20 & 3.92 & 11.98 \\[2ex]
B. ZIOP shifting variables & & 1.82 & 11.61 \\[2ex]
C. ZIOP omitting variables & & & 12.03 \\
\mc{4}{c}{ }\\
\hline
\end{tabular*}
\end{center}
\vspace{-2ex}
\begin{spacing}{1}
\noindent Notes: Entries are $Z$ statistics from Vuong (1989)
specification test comparing non-nested models.
\end{spacing}
\end{footnotesize}
\end{table}

%%%%%%%%%%%%%%%%%%%%%%%%%%%%%%%%%%%%%%%%%%%%%%%%%%%%%%%%%%%%%%%%%%%%%%%
\clearpage

\begin{sidewaystable}
\begin{footnotesize}
\begin{center}
\caption{\label{t:structural_estimates}Structural Estimates of Cost Parameters}
\begin{tabular*}{\textwidth}{l @{\extracolsep{\fill}} S c c S c c S c}
\mc{9}{c}{ } \\
\hline \hline
\mc{9}{c}{ }\\
 & \mc{2}{c}{Type 1 firms} & & \mc{2}{c}{Type 2 firms} & & \mc{2}{c}{Type 3 firms} \\ \cline{2-3} \cline{5-6} \cline{8-9}
\mc{9}{c}{ } \\
 Parameter & \mc{1}{c}{Estimate} & 95\% c.i. & & \mc{1}{c}{Estimate} & 95\% c.i. & & \mc{1}{c}{Estimate} & 95\% c.i. \\
\mc{9}{c}{ } \\
\hline
\mc{9}{c}{ } \\
\mc{9}{l}{Without non-negativity constraint} \\[2ex]
 \hspace{1em}  Cost of monitoring, $\mu_\tau$ & 72.1\one & $[52.8, \quad 104.2]$ & & 48.0\one & $[11.6, \quad 283.2]$ & & 67.7\one & $[37.4, \quad 133.0]$  \\
 \hspace{1em}  Cost of changing price, $\chi_\tau$ & -6.3 & $[-40.5, \quad 4.1]$ & & 42.4 & $[-177.2, \quad 145.8]$ & & -7.4 & $[-67.4, \quad 22.2]$ \\[2ex]
 \hspace{1em}  Valid bootstraps & \mc{2}{c}{100\%} & & \mc{2}{c}{92\%} & & \mc{2}{c}{93\%} \\
\mc{9}{c}{ } \\
\hline
\mc{9}{c}{ } \\
\mc{9}{l}{Imposing non-negativity constraint} \\[2ex]
 \hspace{1em}  Cost of monitoring, $\mu_\tau$ & 68.4\one  & $[50.1, \quad 87.2]$ & & 48.0\ten & $[0.0, \quad 207.8]$ & & 63.0\one & $[37.5, \quad 103.8]$ \\
 \hspace{1em}  Cost of changing price, $\chi_\tau$ & 0.0 & $[0.0, \quad 4.1]$ & & 42.4 & $[0.0, \quad 145.8]$ & & 0.0 & $[0.0, \quad 22.2]$ \\[2ex]
 \hspace{1em}  Valid bootstraps & \mc{2}{c}{100\%} & & \mc{2}{c}{100\%} & & \mc{2}{c}{100\%} \\
\mc{9}{c}{ } \\
\hline
\end{tabular*}
\end{center}
\vspace{-2ex}
\begin{spacing}{1}
\noindent Notes: 95\% confidence intervals computed via bootstrapping
using 200 runs. Bootstrap is taken to be valid if estimated policy
function generates at least 5\% more monitoring episodes than price
changes. Unless a non-negativity constraint is imposed, structural
parameters cannot be independently identified and estimates become
unstable as the number of price changes converges to the number of
monitoring episodes. Statistically significantly different from 0 in a
two-tailed test at the \ten 10\% level, \five 5\% level, \one 1\%
level.
\end{spacing}
\end{footnotesize}
\end{sidewaystable}

%%%%%%%%%%%%%%%%%%%%%%%%%%%%%%%%%%%%%%%%%%%%%%%%%%%%%%%%%%%%%%%%%%%%%%%
\clearpage

\begin{sidewaystable}
\begin{footnotesize}
\begin{center}
\caption{\label{t:robustness}Sensitivity of the Structural Parameters to Adjustments in the Profit Function}
\begin{tabular*}{\textwidth}{l @{\extracolsep{\fill}} S c c S c c S c}
\mc{9}{c}{ } \\
\hline \hline
\mc{9}{c}{ }\\
 & \mc{2}{c}{Type 1 firms} & & \mc{2}{c}{Type 2 firms} & & \mc{2}{c}{Type 3 firms} \\ \cline{2-3} \cline{5-6} \cline{8-9}
\mc{9}{c}{ } \\
 Parameter & \mc{1}{c}{Estimate} & 95\% c.i. & & \mc{1}{c}{Estimate} & 95\% c.i. & & \mc{1}{c}{Estimate} & 95\% c.i. \\
\mc{9}{c}{ } \\
\hline
\mc{9}{c}{ } \\
 \mc{9}{l}{Cost of monitoring, $\mu_\tau$} \\[2ex]
 \hspace{1em} Baseline estimates & 68.4\one & $[50.1, \quad 87.2]$ & & 48.0\ten & $[0.0, \quad 207.8]$ & & 63.0\one & $[37.5, \quad 103.8]$ \\
 \hspace{1em} Adding \$2 to unit cost & 48.1\one & $[33.9, \quad 61.6]$ & & 41.7 & $[0.0, \quad 179.7]$ & & 56.0\one & $[32.6, \quad 92.0]$ \\
 \hspace{1em} Cutting upselling profits in half & 26.0\one & $[15.3, \quad 38.2]$ & & 22.7 & $[0.0, \quad 130.7]$ & & 43.1\one & $[23.5, \quad 69.3]$ \\
 \hspace{1em} Cutting rank 25+ profits in half & 71.8\one & $[53.1, \quad 91.4]$ & & 71.7\one & $[15.4, \quad 282.0]$ & & 90.2\one & $[54.1, \quad 143.4]$ \\
\mc{9}{c}{ } \\
\hline
\mc{9}{c}{ } \\
 \mc{9}{l}{Cost of changing price, $\chi_\tau$} \\[2ex]
 \hspace{1em} Baseline estimates & 0.0 & $[0.0, \quad 4.1]$ & & 42.4 & $[0.0, \quad 145.8]$ & & 0.0 & $[0.0, \quad 22.2]$ \\
 \hspace{1em} Adding \$2 to unit cost & 0.0 & $[0.0, \quad 2.3]$ & & 10.1 & $[0.0, \quad 114.4]$ & & 0.0 & $[0.0, \quad 218.]$ \\
 \hspace{1em} Cutting upselling profits in half & 0.0 & $[0.0, \quad 0.6]$ & & 0.0 & $[0.0, \quad 27.8]$ & & 0.0 & $[0.0, \quad 18.9]$ \\
 \hspace{1em} Cutting rank 25+ profits in half & 0.0 & $[0.0, \quad 3.1]$ & & 74.0 & $[0.0, \quad 188.5]$ & & 0.0 & $[0.0, \quad 17.6]$ \\
\mc{9}{c}{ } \\
\hline
\end{tabular*}
\end{center}
\vspace{-2ex}
\begin{spacing}{1}
\noindent Notes: Baseline estimates are those from
Table~\ref{t:structural_estimates} imposing the non-negativity
constraint. All robustness exercises also impose this
constraint. Statistically significantly different from 0 in a
two-tailed test at the \ten 10\% level, \five 5\% level, \one 1\%
level.
\end{spacing}
\end{footnotesize}
\end{sidewaystable}

%%%%%%%%%%%%%%%%%%%%%%%%%%%%%%%%%%%%%%%%%%%%%%%%%%%%%%%%%%%%%%%%%%%%%%%
\clearpage

\begin{figure}[p]
\centering
\caption{\label{f:pricewatch}Example Pricewatch Webpage} 
\vspace{3ex}
\includegraphics[width=.9\textwidth, viewport=39 141 500 568,clip]{pricewatch.eps}\\[3ex]
\parbox{\textwidth}{\footnotesize {\em Note:} Page downloaded October
  12, 2002.}
\end{figure}

%%%%%%%%%%%%%%%%%%%%%%%%%%%%%%%%%%%%%%%%%%%%%%%%%%%%%%%%%%%%%%%%%%%%%%%
\clearpage

\begin{figure}
\centering
\caption{\label{f:series}Price and Rank Series for Representative Firms of Each Type}
\vspace{5ex}
\includegraphics[width=.52\textwidth, angle=270, viewport=92 30 433 687,clip]{series.eps}\\[5ex]
\parbox{\textwidth}{\footnotesize {\em Note:} Series start later for
  type~2 and type~3 firms because they entered during the month.}
\end{figure}

%%%%%%%%%%%%%%%%%%%%%%%%%%%%%%%%%%%%%%%%%%%%%%%%%%%%%%%%%%%%%%%%%%%%%%%
\clearpage

\begin{figure}[p]
\centering
\caption{\label{f:rank}Effect of Rank on Retailer Outcomes} 
\vspace{3ex}
\includegraphics[width=.7\textwidth, viewport=68 131 446 623,clip]{rank.eps}\\[3ex]
\parbox{\textwidth}{\footnotesize {\em Notes:} Panel A is derived from
  Ellison and Ellison's (2009a) demand estimates, based on their
  regression of the natural log of quantity on linear
  rank. Exponentiating yields the equation graphed, $Q(\Rank_{it}) =
  4.56(1+\Rank_{it})^{-1.29}$. Panel B is derived from Ellison and
  Ellison's (2009a) estimates of upselling profit. They estimate that
  a retailer sells an additional $0.97(1+\Rank_{it})^{-0.77}$ units of
  a medium-quality product with an average markup of \$15.69 and an
  additional $0.49(1+\Rank_{it})^{-0.51}$ units of a high-quality
  product with an average markup of \$31.45. The sum of these, for
  total upselling profit of $U(\Rank_{it}) =
  15.18(1+\Rank_{it})^{-0.77} + 15.48(1+\Rank_{it})^{-0.51}$, the
  function graphed.}
\end{figure}

%%%%%%%%%%%%%%%%%%%%%%%%%%%%%%%%%%%%%%%%%%%%%%%%%%%%%%%%%%%%%%%%%%%%%%%
\clearpage

\begin{figure}[p]
\centering
\caption{\label{f:distributions}Distribution of Size and Spell of Price Changes} 
\vspace{3ex}
\includegraphics[width=.95\textwidth, viewport=6 84 537 679,clip]{distributions.eps}\\[3ex]
\parbox{\textwidth}{\footnotesize {\em Notes:} For legibility,
  histograms in panel~A omit bin for zero price change. When this bar
  is included, the proportions for each firm type sum to 1. Horizontal
  axis truncated at $\pm 10$ price changes. Densities in panel B
  estimated using Epanechnikov kernel. Cleveland's (1979) locally
  weighted regression smoothing (LOWESS) estimated in panel~C. A
  single outlying observation with aspell of over 500 dropped in the
  graph for type~3 firms. Panel~B and~C estimated in Stata~14 using
  default bandwidths.}
\end{figure}

%%%%%%%%%%%%%%%%%%%%%%%%%%%%%%%%%%%%%%%%%%%%%%%%%%%%%%%%%%%%%%%%%%%%%%%
\clearpage

\begin{figure}[p]
\centering
\caption{\label{f:time}Price-Changing Activity During the Day for Retailers on Different Coasts} 
\vspace{3ex}
\includegraphics[width=.85\textwidth, viewport=46 231 483 474,clip]{time.eps}\\[3ex]
\parbox{\textwidth}{\footnotesize {\em Notes:} Graphs show residual
  probability of price change each hour estimated after partialling
  out other covariates. So that the probability functions integrate to
  1, they have been converted into conditional probabilities, i.e.,
  conditional on a price change occurring during the
  day. Probabilities computed from estimates from a maximum likelihood
  model similar to that reported in Table~\ref{t:policy} but with
  $\Night$ indicator replaced by suite of indicators for Eastern Time
  hour and interactions between this suite and an indicator for
  whether the supplier is located on the West Coast. Model estimated
  on subsample of East and West Coast suppliers only.}
\end{figure}

%%%%%%%%%%%%%%%%%%%%%%%%%%%%%%%%%%%%%%%%%%%%%%%%%%%%%%%%%%%%%%%%%%%%%%%
\clearpage

\begin{figure}
\centering
\caption{\label{f:fit}Goodness of Fit of Estimated Policy Function by Firm Type}
\vspace{5ex}
\includegraphics[width=.6\textwidth, angle=270, viewport=62 10 478 704,clip]{fit.eps}\\[5ex]
\parbox{\textwidth}{\footnotesize {\em Note:} Variables on vertical
  axis are discounted using same 0.95 annual factor used to compute
  value functions in the structural estimation. Over the short
  720-hour horizon, discounting has a negligible effect on the
  graphs.}
\end{figure}

%%%%%%%%%%%%%%%%%%%%%%%%%%%%%%%%%%%%%%%%%%%%%%%%%%%%%%%%%%%%%%%%%%%%%%%
\clearpage

\begin{figure}
\centering
\caption{\label{f:states}Goodness of Fit of Estimated Policy Function Across Various Initial States}
\vspace{5ex}
\includegraphics[width=.5\textwidth, angle=270, viewport=113 39 431 670,clip]{states.eps}\\[5ex]
\parbox{\textwidth}{\footnotesize {\em Note:} To save space, graphs
  show results only for type~1 firms.  Monetary profit involves just
  the first two components of $\pi_{it}$, $\Base_{it} + \Upsell_{it}$,
  not the managerial costs structurally estimated later.  Monetary
  profit for the actual sample is also estimated but estimated based
  on firms' actual prices as opposed to the simulations, which use
  prices from the estimated policy function. Na\"{i}ve forecast
  assumes firm earns same profit in each of the 720 hours as in the
  first.  Profits discounted using same 0.95 annual factor used to
  compute value functions in the structural estimation.  Over the
  short 720-hour horizon, discounting has a negligible effect on the
  graphs.}
\end{figure}

%%%%%%%%%%%%%%%%%%%%%%%%%%%%%%%%%%%%%%%%%%%%%%%%%%%%%%%%%%%%%%%%%%%%%%%
\clearpage

\begin{figure}
\centering
\caption{\label{f:counterfactuals}Counterfactual Scenarios with Various Different Managerial Costs}
\vspace{5ex}
\includegraphics[width=.5\textwidth, angle=270, viewport=107 39 437 687,clip]{counterfactuals.eps}\\[5ex]
\parbox{\textwidth}{\footnotesize {\em Note:} Counterfactuals show
  firm's response to a shock to its managerial costs. Dots indicate
  counterfactuals associated with optimal policies for managerial
  costs set to the structural estimates.  To save space, graphs show
  results only for type~1 firms.  Net profit subtracts managerial
  costs from $\pi_{it}$. Profits discounted using same 0.95 annual
  factor used to compute value functions in the structural
  estimation. Over the short 720-hour horizon, discounting has a
  negligible effect on the graphs.}
\end{figure}

\end{document}











